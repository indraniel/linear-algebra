% Chapter 2, Topic _Linear Algebra_ Jim Hefferon
%  http://joshua.smcvt.edu/linearalgebra
%  2001-Jun-11
\topic{Fields}
\index{field|(}
Computations involving only integers or only rational numbers are much
easier than those with real numbers.
Could other algebraic structures, such as the integers or the rationals,
work in the
place of \( \Re \) in the definition of a vector space?

If we take ``work'' to mean that the results of this chapter
remain true
then there is a natural list of conditions that a structure 
(that is, number system)
must have in order to work in the place of $\Re$.
A \definend{field}\index{field!definition} is a set 
\( \F \) with operations
`\( + \)'
and `\( \cdot \)' such that
\begin{tfae}
   \item for any \( a,b\in\F \) the result of \( a+b \) is in \( \F \), and
         \( a+b=b+a \), and
         if \( c\in\F \) then \( a+(b+c)=(a+b)+c \)
   \item for any \( a,b\in\F \) the result of \( a\cdot b \) is in \( \F \), and
         \( a\cdot b=b\cdot a \), and
         if \( c\in\F \) then \( a\cdot (b\cdot c)=(a\cdot b)\cdot c \)
   \item if \( a,b,c\in\F \) then \( a\cdot (b+c)=a\cdot b+a\cdot c \)
   \item there is an element \( 0\in\F \) such that
         if \( a\in\F \) then \( a+0=a \), and
         for each \( a\in\F \) there is an element \( -a\in\F \)
                such that \( (-a)+a=0 \)
   \item there is an element \( 1\in\F \) such that
         if \( a\in\F \) then \( a\cdot 1=a \), and
         for each element \( a\neq 0 \) of \( \F \)
                there is an element \( a^{-1}\in\F \)
                such that \( a^{-1}\cdot a=1 \).
\end{tfae}

For example, the algebraic structure 
consisting of the set of real numbers along with its usual
addition and multiplication operation is a field.
Another field is the set of rational numbers with its usual addition
and multiplication operations.
An example of an algebraic structure that is not a field is
the integers, because it fails the final condition.

Some examples are more surprising.
The set \( \mathbb{B}=\set{0,1} \) under these operations:
\begin{center}
  \begin{tabular}{c|cc}
    \( + \) &\( 0 \) &\( 1 \) \\
    \hline
    \( 0 \) &\( 0 \) &\( 1 \) \\
    \( 1 \) &\( 1 \) &\( 0 \)
  \end{tabular}
  \qquad
  \begin{tabular}{c|cc}
    \( \cdot \) &\( 0 \) &\( 1 \) \\
     \hline
       \( 0 \)  &\( 0 \) &\( 0 \)  \\
       \( 1 \)  &\( 0 \) &\( 1 \)
  \end{tabular}
\end{center}
is a field; see \nearbyexercise{exer:BinField}.

We could in this book develop Linear Algebra as the theory of
vector spaces with scalars from an arbitrary field.
In that case, 
almost all of the statements here would carry over by replacing
`\( \Re \)' with `\( \F \)', that is, by
taking coefficients, vector entries,
and matrix entries to be elements of \( \F \)
(the exceptions are statements involving distances or angles,
which would need additional development).
Here are some examples; each applies to a vector space \( V \)
over a field \( \F \).
\begin{itemize}
  %\makebox[\parindent]{\ \hfil\ }
  \item[$*$] For any \( \vec{v}\in V \) and \( a\in\F \),
  (i)~\( 0\cdot\vec{v}=\zero \),
  (ii)~\( -1\cdot\vec{v}+\vec{v}=\zero \),
  and (iii)~\( a\cdot\vec{0}=\vec{0} \).

  \item[$*$] The span, the set of linear combinations, of a subset of \( V \)
  is a subspace of \( V \).

  \item[$*$] Any subset of a linearly independent set is also 
     linearly independent.

  \item[$*$] In a finite-dimensional vector space, 
    any two bases have the same number of elements.
\end{itemize}
(Even statements that don't explicitly mention \( \F \) use
field properties in their proof.)

We will not develop vector spaces in this more general setting because
the additional abstraction can be a distraction.
The ideas we want to bring out already appear when we stick to the reals.

The exception is Chapter Five.
There we must factor polynomials,
so we will switch to considering vector spaces over the
field of complex numbers.

\begin{exercises}
  \item 
    Check that the real numbers form a field.
    \begin{answer}
      Going through the five conditions shows that they are all familiar from 
      elementary mathematics.
    \end{answer}
  \item 
    Prove that these are fields.
    \begin{exparts*}
       \partsitem The rational numbers $\Q$
       \partsitem The complex numbers  $\C$
    \end{exparts*}
    \begin{answer}
      As with the prior question, going through the five conditions 
      shows that
      for both of these structures, 
      the properties are familiar.
    \end{answer}
  \item 
     Give an example that shows that the integer number system
     is not a field.
     \begin{answer}
       The integers fail condition~(5).
       For instance, there is no multiplicative inverse for $2$\Dash while
       $2$ is an integer, $1/2$ is not.
     \end{answer}
  \item \label{exer:BinField} 
     Check that the set $\mathbb{B}=\set{0,1}$ is a field under the operations 
     listed above, 
     \begin{answer}
       We can do these checks by listing all of the possibilities.
       For instance, to verify the first half of condition~(2) we must check
       that the structure is closed under addition and that addition
       is commutative  $a+b=b+a$,
       we can check both of these 
       for all possible pairs $a$ and~$b$ because there
       are only four such pairs.  
       Similarly, for associativity, there are only eight triples $a$, $b$,
       $c$, and so the check is not too long.
       (There are other ways to do the checks; in particular, you may 
       recognize these operations as arithmetic modulo~$2$.
       But an exhaustive check is not onerous)
     \end{answer}
  \item 
     Give suitable operations to make the set $\set{0,1,2}$
     a field.     
     \begin{answer}
       These will do.
       \begin{center}
          \begin{tabular}{c|ccc}
             \( + \) &\( 0 \) &\( 1 \) &$2$ \\
             \hline
             \( 0 \) &\( 0 \) &\( 1 \) &$2$ \\
             \( 1 \) &\( 1 \) &\( 2 \) &$0$ \\
             $2$     &$2$     &$0$     &$1$
          \end{tabular}
          \qquad
          \begin{tabular}{c|ccc}
            \( \cdot \) &\( 0 \) &\( 1 \) &$2$ \\
            \hline
            \( 0 \)  &\( 0 \) &\( 0 \) &$0$ \\
            \( 1 \)  &\( 0 \) &\( 1 \) &$2$ \\
            $2$      &$0$     &$2$     &$1$
          \end{tabular}
       \end{center}
       As in the prior item, we could 
       verify that they satisfy the conditions  
       by listing all of the cases.
     \end{answer}
\end{exercises}
\index{field|)}
%\include{la}
%\end{document}
