% see: https://groups.google.com/forum/?fromgroups#!topic/comp.text.tex/s6z9Ult_zds
\makeatletter\let\ifGm@compatii\relax\makeatother 
\documentclass[10pt,t,serif,professionalfont]{beamer}
\PassOptionsToPackage{pdfpagemode=FullScreen}{hyperref}
\PassOptionsToPackage{usenames,dvipsnames}{color}
% \DeclareGraphicsRule{*}{mps}{*}{}
\usepackage{../linalgjh}
\usepackage{present}
\usepackage{xr}\externaldocument{../map2} % read refs from .aux file
\usepackage{xr}\externaldocument{../vs3} % read refs from .aux file
\usepackage{catchfilebetweentags}
\usepackage{etoolbox} % from http://tex.stackexchange.com/questions/40699/input-only-part-of-a-file-using-catchfilebetweentags-package
\makeatletter
\patchcmd{\CatchFBT@Fin@l}{\endlinechar\m@ne}{}
  {}{\typeout{Unsuccessful patch!}}
\makeatother

\mode<presentation>
{
  \usetheme{boxes}
  \setbeamercovered{invisible}
  \setbeamertemplate{navigation symbols}{} 
}
\addheadbox{filler}{\ }  % create extra space at top of slide 
\hypersetup{colorlinks=true,linkcolor=blue} 

\title[Homomorphisms] % (optional, use only with long paper titles)
{Three.II Homomorphisms}

\author{\textit{Linear Algebra} \\ {\small Jim Hef{}feron}}
\institute{
  \texttt{http://joshua.smcvt.edu/linearalgebra}
}
\date{}


\subject{Homomorphisms}
% This is only inserted into the PDF information catalog. Can be left
% out. 

\begin{document}
\begin{frame}
  \titlepage
\end{frame}

% =============================================
% \begin{frame}{Reduced Echelon Form} 
% \end{frame}



% ..... Three.II.1 .....
\section{Definition}
%..........
\begin{frame}{Homomorphism}
\df[def:Homo]
\ExecuteMetaData[../map2.tex]{df:Homo}
\end{frame}




%..........
\begin{frame}
We proved these in the context of studying isomorphisms.

\lm[le:HomoSendsZeroToZero]
\ExecuteMetaData[../map2.tex]{lm:HomoSendsZeroToZero}

\pause
\lm[le:HomoPreserveLinCombo]
\ExecuteMetaData[../map2.tex]{lm:HomoPreserveLinCombo}
\end{frame}




%..........
\begin{frame}
\th[th:HomoDetActOnBasis]
\ExecuteMetaData[../map2.tex]{th:HomoDetActOnBasis}

\pause
\pf
\ExecuteMetaData[../map2.tex]{pf:HomoDetActOnBasis0}

\pause
\ExecuteMetaData[../map2.tex]{pf:HomoDetActOnBasis1}
\end{frame}
\begin{frame}
\ExecuteMetaData[../map2.tex]{pf:HomoDetActOnBasis2}
\qed
\end{frame}




%..........
\begin{frame}
\df[df:LinearTransformation]
\ExecuteMetaData[../map2.tex]{df:LinearTransformation}
\end{frame}




%..........
\begin{frame}
\lm[le:SpLinFcns]
\ExecuteMetaData[../map2.tex]{lm:SpLinFcns}

\ExecuteMetaData[../map2.tex]{SpLinFcns}

\pause
\pf
\ExecuteMetaData[../map2.tex]{pf:SpLinFcns}
\qed
\end{frame}



% ..... Three.II.2 .....
\section{Range space and null space}
%..........
\begin{frame}
\lm[le:RangeIsSubSp]
\ExecuteMetaData[../map2.tex]{lm:RangeIsSubSp}

\pause
\pf
\ExecuteMetaData[../map2.tex]{pf:RangeIsSubSp}
\qed
\end{frame}




%..........
\begin{frame}{Range space}
\df[df:RangeSpace]
\ExecuteMetaData[../map2.tex]{df:RangeSpace}
\end{frame}




%..........
\begin{frame}
\lm[le:NullspIsSubSp]
\ExecuteMetaData[../map2.tex]{lm:NullspIsSubSp}

\pause
\pf
\ExecuteMetaData[../map2.tex]{pf:NullspIsSubSp}
\qed
\end{frame}




%..........
\begin{frame}{Null space}
\df[df:NullSpace]
\ExecuteMetaData[../map2.tex]{df:NullSpace}
\end{frame}




%..........
\begin{frame}{Rank plus nullity}
\th[th:RankPlusNullEqDim]
\ExecuteMetaData[../map2.tex]{th:RankPlusNullEqDim}

\pause
\pf
\ExecuteMetaData[../map2.tex]{pf:RankPlusNullEqDim0}

\pause
\ExecuteMetaData[../map2.tex]{pf:RankPlusNullEqDim1}
\end{frame}
\begin{frame}
\ExecuteMetaData[../map2.tex]{pf:RankPlusNullEqDim2}
\qed
\end{frame}




%..........
\begin{frame}
\lm[lm:ImageLinearlyDependentIsLinearlyDependent]
\ExecuteMetaData[../map2.tex]{lm:ImageLinearlyDependentIsLinearlyDependent}

\pause
\pf
\ExecuteMetaData[../map2.tex]{pf:ImageLinearlyDependentIsLinearlyDependent}
\qed
\end{frame}




%..........
\begin{frame}
\th[th:OOHomoEquivalence]
\ExecuteMetaData[../map2.tex]{th:OOHomoEquivalence}

\pause
\pf
\ExecuteMetaData[../map2.tex]{pf:OOHomoEquivalence0}
\end{frame}
\begin{frame}
\ExecuteMetaData[../map2.tex]{pf:OOHomoEquivalence1}

\pause
\ExecuteMetaData[../map2.tex]{pf:OOHomoEquivalence2}
\end{frame}
\begin{frame}
\ExecuteMetaData[../map2.tex]{pf:OOHomoEquivalence3}
\qed
\end{frame}




%...........................
% \begin{frame}
% \ExecuteMetaData[../gr3.tex]{GaussJordanReduction}
% \df[def:RedEchForm]
% 
% \end{frame}
\end{document}
