% see: https://groups.google.com/forum/?fromgroups#!topic/comp.text.tex/s6z9Ult_zds
\makeatletter\let\ifGm@compatii\relax\makeatother 
\documentclass[10pt,t]{beamer}
\usefonttheme{professionalfonts}
\usefonttheme{serif}
\PassOptionsToPackage{pdfpagemode=FullScreen}{hyperref}
\PassOptionsToPackage{usenames,dvipsnames}{color}
% \DeclareGraphicsRule{*}{mps}{*}{}
\usepackage{../linalgjh}
\usepackage{present}
\usepackage{xr}\externaldocument{../map1} % read refs from .aux file
\usepackage{catchfilebetweentags}
\usepackage{etoolbox} % from http://tex.stackexchange.com/questions/40699/input-only-part-of-a-file-using-catchfilebetweentags-package
\makeatletter
\patchcmd{\CatchFBT@Fin@l}{\endlinechar\m@ne}{}
  {}{\typeout{Unsuccessful patch!}}
\makeatother

\mode<presentation>
{
  \usetheme{boxes}
  \setbeamercovered{invisible}
  \setbeamertemplate{navigation symbols}{} 
}
\addheadbox{filler}{\ }  % create extra space at top of slide 
\hypersetup{colorlinks=true,linkcolor=blue} 

\title[Isomorphisms] % (optional, use only with long paper titles)
{Three.I Isomorphisms}

\author{\textit{Linear Algebra} \\ {\small Jim Hef{}feron}}
\institute{
  \texttt{http://joshua.smcvt.edu/linearalgebra}
}
\date{}


\subject{Isomorphisms}
% This is only inserted into the PDF information catalog. Can be left
% out. 

\begin{document}
\begin{frame}
  \titlepage
\end{frame}

% =============================================
% \begin{frame}{Reduced Echelon Form} 
% \end{frame}



% ..... Three.I.1 .....
\section{Definition and examples}




%..........
\begin{frame}
\ex
People often have the intuition that the two vector spaces 
$\Re^2$ and $\polyspace_1$ are ``the same,'' for instance in that 
\begin{align*}
    &\colvec{1 \\ 2}
    \quad\text{is just like}\quad
    1+2x                            \\
 \text{and} &\colvec{-3 \\ 1/2}
    \quad\text{is just like}\quad
    -3-(1/2)x                           
\end{align*}
etc.
What makes the one ``just like'' the other is that this association
holds through the operations of addition
\begin{multline*}
  \colvec{1 \\ 2}+\colvec{-3 \\ 1/2}=\colvec{-2 \\ 5/2}  \\
  \text{is just like}\quad
  (1+2x)+(-3+(1/2)x)=-2+(5/2)x
\end{multline*}
and scalar multiplication.
\begin{equation*}
  3\colvec{1 \\ 2}=\colvec{3 \\ 6}
  \quad\text{is just like}\quad
  3(1+2x)=3+6x
\end{equation*}
\end{frame}\begin{frame}
More generally,
we can express the same-ness of the spaces by 
associating each two-tall vector with a linear polynomial.
\begin{equation*}
  \colvec{a \\ b}
  \quad\longleftrightarrow\quad
  a+bx
\end{equation*}
such that the association holds through the vector space operations of
addition
\begin{multline*}
  \colvec{a_1 \\ b_1}+\colvec{a_2 \\ b_2}=\colvec{a_1+a_2 \\ b_1+b_2}    \\
  \quad\longleftrightarrow\quad
  (a_1+b_1x)+(a_2+b_2x)=(a_1+a_2)+(b_1+b_2)x
\end{multline*}
and scalar multiplication.
\begin{equation*}
  r\colvec{a \\ b}=\colvec{ra \\ rb}
  \quad\longleftrightarrow\quad
  r(a+bx)=(ra)+(rb)x
\end{equation*}
We say that the association \definend{preserves the structure} of the spaces.
\end{frame}




%..........
\begin{frame}
\ex
We can think of $\matspace_{\nbyn{2}}$ as ``the same'' as $\Re^4$
if we associate in this way.
\begin{equation*}
  \begin{mat}
    a  &b  \\
    c  &d
  \end{mat}
  \quad\longleftrightarrow\quad
  \colvec{a \\ b \\ c \\ d}
\end{equation*}
For instance, this association matches these two.
\begin{equation*}
  \begin{mat}[r]
    1  &-1  \\
    2  &-2
  \end{mat}
  \quad\longleftrightarrow\quad
  \colvec[r]{1 \\ -1 \\ 2 \\ -2}
\end{equation*}
\pause
This association holds up under addition.
\begin{multline*}
  \begin{mat}
    a_1  &b_1  \\
    c_1  &d_1
  \end{mat}
  +
  \begin{mat}
    a_2  &b_2  \\
    c_2  &d_2
  \end{mat}
  =
  \begin{mat}
    a_1+a_2  &b_1+b_2  \\
    c_1+c_2  &d_1+d_2
  \end{mat}                                    \\  
  \quad\longleftrightarrow\quad
  \colvec{a_1 \\ b_1 \\ c_1 \\ d_1}
  +
  \colvec{a_2 \\ b_2 \\ c_2 \\ d_2}
  =
  \colvec{a_1+a_2 \\ b_1+b_2 \\ c_1+c_2 \\ d_1+d_2}
\end{multline*}
\end{frame}
\begin{frame}
Here is an example of that with particular vectors.
\begin{multline*}
  \begin{mat}
    1  &-1  \\
    2  &-2
  \end{mat}
  +
  \begin{mat}
    0  &4  \\
    3  &-3
  \end{mat}
  =
  \begin{mat}
    1  &3  \\
    5  &-5
  \end{mat}                                    \\  
  \quad\longleftrightarrow\quad
  \colvec{1 \\ -1 \\ 2 \\ -2}
  +
  \colvec{0 \\ 4 \\ 3 \\ -3}
  =
  \colvec{1 \\ 3 \\ 5 \\ -5}
\end{multline*}
\pause
The association also holds under scalar multiplication.
\begin{equation*}
  r\cdot\begin{mat}
   a  &b  \\
   c  &d 
  \end{mat}
  =
  \begin{mat}
    ra  &rb  \\
    rc  &rd
  \end{mat}
  \quad\longleftrightarrow\quad
  r\cdot\colvec{a \\ b \\ c \\ d}
  =
  \colvec{ra \\ rb \\ rc \\ rd} 
\end{equation*}
This illustrates with particular vectors.
\begin{equation*}
  2\cdot\begin{mat}
   1  &-1  \\
   2  &-2 
  \end{mat}
  =
  \begin{mat}
    2  &-2  \\
    4  &-4
  \end{mat}
  \quad\longleftrightarrow\quad
  2\cdot\colvec{1 \\ -1 \\ 2 \\ -1}
  =
  \colvec{2 \\ -2 \\ 4 \\ -4} 
\end{equation*}
\end{frame}




%..........
\begin{frame}{Isomorphism}
\df[def:Isomorphism]
\ExecuteMetaData[../map1.tex]{df:Isomorphism}
\end{frame}




%..........
\begin{frame}
\ex
The space of quadratic polynomials $\polyspace_2$ is isomorphic to the
space $\Re^3$ under this map.
\begin{equation*}
  f(a_0+a_1x+a_2x^2)=\colvec{a_0 \\ a_1 \\ a_2}
\end{equation*}
For instance, here is the action of $f$ on two inputs.
\begin{equation*}
  f(1+2x+3x^2)=\colvec{1 \\ 2 \\ 3}
  \quad\text{and}\quad
  f(3+4x^2)=\colvec{3 \\ 0 \\ 4}  
\end{equation*}
To verify that $f$ is an isomorphism we must check conditions~(1) and~(2).
\end{frame}
\begin{frame}
The first part of~(1) is that $f$ is one-to-one.
We usually verify this by assuming that the function yields the same output
on two inputs and then show that the two inputs must therefore be equal.
So assume that $f(a_0+a_1x+a_2x^2)=f(b_0+b_1x+b_2x^2)$.
By definition of $f$ we have
\begin{equation*}
  \colvec{a_0 \\ a_1 \\ a_2}
  =
  \colvec{b_0 \\ b_1 \\ b_2}
\end{equation*}
and two column vectors are equal only if their entries are equal
$a_0=b_0$, $a_1=b_1$, and~$a_2=b_2$.
Thus the starting inputs are equal
$a_0+a_1x+a_2x^2=b_0+b_1x+b_2x^2$ and so $f$ is one-to-one.

\pause
The second part of~(1) is that $f$ is onto.
We usually verify this by taking an element of the codomain
and producing an element of the domain that maps to it.
So consider this member of $\Re^3$.
\begin{equation*}
  \colvec{u \\ v \\ w}
\end{equation*}
Observe that it is the image under $f$ of the member 
$u+vx+wx^2$ of the domain.
Thus $f$ is onto.
\end{frame}
\begin{frame}
Condition~(2) also has two halves.
First we must show that $f$ preserves addition.
Consider $f$ acting on the sum of two elements of the domain.
\begin{equation*}
  f(\,(a_0+a_1x+a_2x^2)+(b_0+b_1x+b_2x^2)\,)
  =f((a_0+b_0)+(a_1+b_1)x+(a_2+b_2)x^2)
\end{equation*}
The function maps the linear polynomial on the right to this vector.
\begin{equation*}
  \colvec{a_0+b_0 \\ a_1+b_1 \\ a_2+b_2}
  =\colvec{a_0 \\ a_1 \\ a_2}
  +
  \colvec{b_0 \\ b_1 \\ b_2}
\end{equation*}
The right side is $f(a_0+a_1x+a_2x^2)+f(b_0+b_1x+b_2x^2)$, as required.

\pause
We finish by checking that $f$ preserves scalar multiplication.
\begin{equation*}
  r\cdot f(a_0+a_1x+a_2x^2)
  =r\cdot\colvec{a_0 \\ a_1 \\ a_2}
  =\colvec{ra_0 \\ ra_1 \\ ra_2}
  =f(\,(ra_0)+(ra_1)x+(ra_2)x^2)
\end{equation*}
\qed
\end{frame}




%..........
\begin{frame}{Automorphisms}
\df[df:Automorphism]
\ExecuteMetaData[../map1.tex]{df:Automorphism}

\pause
\ex[exam:RigidPlaneMapsAutos]
\ExecuteMetaData[../map1.tex]{ex:RigidPlaneMapsAutos0}
\centergraphic{../ch3.14}

\pause
\ExecuteMetaData[../map1.tex]{ex:RigidPlaneMapsAutos1}
\centergraphic{../ch3.15}
\end{frame}
\begin{frame}
\ExecuteMetaData[../map1.tex]{ex:RigidPlaneMapsAutos2}
\centergraphic{../ch3.16}

Checking that each of these is an isomorphism is an exercise.
\end{frame}



%..........
\begin{frame}
\lm[le:IsoSendsZeroToZero]
\ExecuteMetaData[../map1.tex]{lm:IsoSendsZeroToZero}

\pause
\pf
\ExecuteMetaData[../map1.tex]{pf:IsoSendsZeroToZero}
\qed
\end{frame}




%..........
\begin{frame}
\lm[le:PresStructIffPresCombos]
\ExecuteMetaData[../map1.tex]{lm:PresStructIffPresCombos}

\pause
\pf
\ExecuteMetaData[../map1.tex]{pf:PresStructIffPresCombos0}

\pause
\ExecuteMetaData[../map1.tex]{pf:PresStructIffPresCombos1}
\end{frame}
\begin{frame}
\ExecuteMetaData[../map1.tex]{pf:PresStructIffPresCombos2}
\qed
\end{frame}




%..........
\begin{frame}
That result shortens the checking that a function preserves
the structure of a vector space.   

\ex
This line through the origin
\begin{equation*}
  L=\set{t\cdot\colvec{1 \\ 2} \suchthat t\in\Re}
\end{equation*}
is a vector space under the addition and scalar multiplication operations
that it inherits from $\Re^2$.
\begin{equation*}
  t_1\cdot\colvec{1 \\ 2}+t_2\cdot\colvec{1 \\ 2}=(t_1+t_2)\cdot\colvec{1 \\ 2}
  \qquad
  r\big(t\cdot\colvec{1 \\ 2}\big)=(rt)\cdot\colvec{1 \\ 2}
\end{equation*}
This space is isomorphic to $\Re^1$ under this map.
\begin{equation*}
  f(\,t\cdot\colvec{1 \\ 2}\,)=t
\end{equation*}
\end{frame}
\begin{frame}
We first verify that $f$ is one-to-one.
Suppose that $f$ maps two members of $L$ to the same output.
\begin{equation*}
  f(\,t_1\cdot\colvec{1 \\ 2}\,)=f(\,t_2\cdot\colvec{1 \\ 2}\,)
\end{equation*}
Then by definition of the function $f$ we have that $t_1=t_2$ and so the two
members of $L$ are equal.

\pause 
Next we check that $f$ is an onto map.
Consider this member of the codomain: $r\in\Re$.
There is a member of the domain that maps to it, namely this member of $L$.
\begin{equation*}
  f(\,r\cdot\colvec{1 \\ 2}\,)
\end{equation*}
% Thus $f$ is onto.

\pause
To finish, we use the lemma to check that $f$ preserves structure.
\begin{equation*}
  f(\,t_1\cdot\colvec{1 \\ 2}+t_2\cdot\colvec{1 \\ 2}\,)
  =f(\,\colvec{t_1+t_2 \\ 2(t_1+t_2)}\,)
  =t_1+t_2
  =
  f(\,t_1\cdot\colvec{1 \\ 2}\,)+f(\,t_2\cdot\colvec{1 \\ 2}\,)
\end{equation*}

\end{frame}



% ..... Three.I.2 .....
\section{Dimension characterizes isomorphism}
%..........
\begin{frame}
\lm[lem:IsoInvAlsoIso]
\ExecuteMetaData[../map1.tex]{lm:IsoInvAlsoIso}

\pause
\pf
\ExecuteMetaData[../map1.tex]{pf:IsoInvAlsoIso0}

\pause
\ExecuteMetaData[../map1.tex]{pf:IsoInvAlsoIso1}
\qed
\end{frame}




%..........
\begin{frame}
\th[th:IsoEquivRel]
\ExecuteMetaData[../map1.tex]{th:IsoEquivRel}

\pause
\pf
\ExecuteMetaData[../map1.tex]{pf:IsoEquivRel0}

\pause
\ExecuteMetaData[../map1.tex]{pf:IsoEquivRel1}
\end{frame}
\begin{frame}
\ExecuteMetaData[../map1.tex]{pf:IsoEquivRel2}
\qed
\end{frame}




%..........
\begin{frame}
Thus the collection of all finite-dimensional vector spaces
of partitioned into classes.
Two spaces are in the same class if they are isomorphic.
\centergraphic{../ch3.17}

The next result characterizes the classes.
\end{frame}




%..........
\begin{frame}
\th[th:NDimSpaceIsoRN]
\ExecuteMetaData[../map1.tex]{th:NDimSpaceIsoRN}

\medskip
The proof is the next two lemmas.
\medskip

\pause
\lm[lem:IsoImpliesSameDim]
\ExecuteMetaData[../map1.tex]{lm:IsoImpliesSameDim}

\pause
\pf
\ExecuteMetaData[../map1.tex]{pf:IsoImpliesSameDim0}
\end{frame}
\begin{frame}
\ExecuteMetaData[../map1.tex]{pf:IsoImpliesSameDim1}
\qed
\end{frame}




%..........
\begin{frame}
\lm[lem:EqDimImpIso]
\ExecuteMetaData[../map1.tex]{lm:EqDimImpIso}

\pause
\pf
\ExecuteMetaData[../map1.tex]{pf:EqDimImpIso0}

\pause
\ExecuteMetaData[../map1.tex]{pf:EqDimImpIso1}
\end{frame}
\begin{frame}
\ExecuteMetaData[../map1.tex]{pf:EqDimImpIso2}

\pause
\ExecuteMetaData[../map1.tex]{pf:EqDimImpIso3}
\end{frame}
\begin{frame}
\ExecuteMetaData[../map1.tex]{pf:EqDimImpIso4}
\qed

\pause
\medskip
\no
The second paragraph's 
representation map $\text{Rep}_B$ is a well-defined function since
every vector $\vec{v}$ has a unique representation, with respect to
a particular basis.
\end{frame}




%..........
\begin{frame}
\ex
The plane $2x-y+z=0$ through the origin in $\Re^3$ is a vector space.
Considering that a one-equation linear system
and paramatrizing with the free variables
\begin{equation*}
  P=\set{\colvec{x \\ y \\ z}
         =\colvec{1/2 \\ 1 \\ 0}y
          +\colvec{1/2 \\ 0 \\ -1}z
         \suchthat y,z\in\Re}
\end{equation*}
gives a basis.
\begin{equation*}
  B=\sequence{\colvec{1/2 \\ 1 \\ 0},
              \colvec{1/2 \\ 0 \\ -1}}
\end{equation*}
This is a dimension~$2$ space. 
For instance, it is isomorphic to $\Re^2$.
\end{frame}




%..........
\begin{frame}
\co[co:FiniteDimensionalIsoToReN]
\ExecuteMetaData[../map1.tex]{co:FiniteDimensionalIsoToReN}

\pause
\medskip
Thus the real spaces $\Re^n$ form a set of canonical
representatives of the isomorphism classes.
\centergraphic{../ch3.18}
\end{frame}



%...........................
% \begin{frame}
% \ExecuteMetaData[../gr3.tex]{GaussJordanReduction}
% \df[def:RedEchForm]
% 
% \end{frame}
\end{document}
