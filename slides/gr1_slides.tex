% see: https://groups.google.com/forum/?fromgroups#!topic/comp.text.tex/s6z9Ult_zds
\makeatletter\let\ifGm@compatii\relax\makeatother 
\documentclass[10pt,t,serif,professionalfont]{beamer}
\PassOptionsToPackage{pdfpagemode=FullScreen}{hyperref}
\PassOptionsToPackage{usenames,dvipsnames}{color}
% \usepackage{linear_algebra}
\usepackage{../linalgjh}
\usepackage{present}
\usepackage{catchfilebetweentags}
\usepackage{etoolbox} % from http://tex.stackexchange.com/questions/40699/input-only-part-of-a-file-using-catchfilebetweentags-package
\makeatletter
\patchcmd{\CatchFBT@Fin@l}{\endlinechar\m@ne}{}
  {}{\typeout{Unsuccessful patch!}}
\makeatother

\mode<presentation>
{
  \usetheme{boxes}
  \setbeamercovered{invisible}
  \setbeamertemplate{navigation symbols}{} 
}
\addheadbox{filler}{\ }  % create extra space at top of slide 
\hypersetup{colorlinks=true,linkcolor=blue} 

\title[Solving Linear Systems] % (optional, use only with long paper titles)
{Solving Linear Systems}

\author{\textit{Linear Algebra} \\ {\small Jim Hef{}feron}}
\institute{
  \texttt{http://joshua.smcvt.edu/linearalgebra}
}
\date{}
% {\def\protect{\global\let\YesIMeanThisRef}\ref}
% \AtBeginDocument{\def\ref{\protect\YesIMeanThisRef}}
\usepackage{xparse}% http://ctan.org/pkg/xparse

\makeatletter
\NewDocumentCommand{\reff}{s m}{%
  \IfBooleanTF{#1}% Check for starred variant
    {\beamer@origref{#2}}% \reff*
    {\hyperlink{#2}{\beamer@origref{#2}}}% \reff
}
\makeatother


\subject{Solving Linear Systems}
% This is only inserted into the PDF information catalog. Can be left
% out. 
% \AtBeginDocument{\makeatletter
% \let\ref\@refstar
% \makeatother}

\begin{document}
\begin{frame}
  \titlepage
\end{frame}

% =============================================



% ..... One.I.1 .....
\section{Gauss' method}
\begin{frame}{Linear systems} 
\dfn[\reff*{def:linearcombination}]
\ExecuteMetaData[../gr1.tex]{def:linearcombination}
% \booktheorem{def:linearcombination}
\end{frame}


%...........................
\begin{frame}
\dfn[\reff*{def:linearcombination}]
\ExecuteMetaData[../gr1.tex]{def:linearequations}
  
\end{frame}



%...........................
\begin{frame}{Gauss' method}
\thm
\ExecuteMetaData[../gr1.tex]{th:gaussmethod}

\end{frame}



%...........................
\begin{frame}
\pf
\ExecuteMetaData[../gr1.tex]{pf:gaussmethod}
\dfn{1.2}
\ExecuteMetaData[../gr1.tex]{def:gaussmethod}

\end{frame}








% ..... One.I.2 .....
\section{Describing the solution set}
\begin{frame}
\frametitle{Classifying solution sets} 

\end{frame}



% ..... One.I.3 .....
\section{$\text{General}=\text{Particular}+\text{Homogeneous}$}
\begin{frame}
\frametitle{Form of solution sets} 

\end{frame}








%...........................
% \begin{frame}
% 
% \end{frame}
\end{document}
