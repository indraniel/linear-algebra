% see: https://groups.google.com/forum/?fromgroups#!topic/comp.text.tex/s6z9Ult_zds
\makeatletter\let\ifGm@compatii\relax\makeatother 
\documentclass[10pt,t,serif,professionalfont]{beamer}
\PassOptionsToPackage{pdfpagemode=FullScreen}{hyperref}
\PassOptionsToPackage{usenames,dvipsnames}{color}
% \DeclareGraphicsRule{*}{mps}{*}{}
\usepackage{../linalgjh}
\usepackage{present}
\usepackage{xr}\externaldocument{../det1} % read refs from .aux file
\usepackage{xr}\externaldocument{../map4} % read refs from .aux file
\usepackage{catchfilebetweentags}
\usepackage{etoolbox} % from http://tex.stackexchange.com/questions/40699/input-only-part-of-a-file-using-catchfilebetweentags-package
\makeatletter
\patchcmd{\CatchFBT@Fin@l}{\endlinechar\m@ne}{}
  {}{\typeout{Unsuccessful patch!}}
\makeatother

\mode<presentation>
{
  \usetheme{boxes}
  \setbeamercovered{invisible}
  \setbeamertemplate{navigation symbols}{} 
}
\addheadbox{filler}{\ }  % create extra space at top of slide 
\hypersetup{colorlinks=true,linkcolor=blue} 

\title[Determinants] % (optional, use only with long paper titles)
{Four.I Determinants; Definition}

\author{\textit{Linear Algebra} \\ {\small Jim Hef{}feron}}
\institute{
  \texttt{http://joshua.smcvt.edu/linearalgebra}
}
\date{}


\subject{Determinants}
% This is only inserted into the PDF information catalog. Can be left
% out. 

\begin{document}
\begin{frame}
  \titlepage
\end{frame}

% =============================================
% \begin{frame}{Reduced Echelon Form} 
% \end{frame}



% ..... Four.I.1,2 .....
\section{Properties of Determinants}
%..........
\begin{frame}{Nonsingular matrices}
An \( \nbyn{n} \) matrix \( T \) is nonsingular if and only if
each of these holds:
\ExecuteMetaData[../det1.tex]{EquivalentOfNonsingular}
In this chapter we will give a formula that determines whether a
matrix is nonsingular.
\end{frame}




\begin{frame}
\ExecuteMetaData[../det1.tex]{DeterminantIntro}  
\end{frame}




\begin{frame}
We will define the determinant function not by listing how 
to compute it (because that has no clear connection to its purpose)
but instead by some of its properties.
Then we will show that only one function with those properties exists.
\end{frame}




\begin{frame}{Definition of determinant}
\df[def:Det]
\ExecuteMetaData[../det1.tex]{df:Det}

\pause 
\re[rem:SwapRowsRedun] 
\ExecuteMetaData[../det1.tex]{re:SwapRowsRedun}
\end{frame}




\begin{frame}{Consequences of the definition}
\lm[le:IdenRowsDetZero]
\ExecuteMetaData[../det1.tex]{lm:IdenRowsDetZero}

\pause 
\pf 
\ExecuteMetaData[../det1.tex]{pf:IdenRowsDetZero0}

\pause
\ExecuteMetaData[../det1.tex]{pf:IdenRowsDetZero1}
\end{frame}
\begin{frame}
\ExecuteMetaData[../det1.tex]{pf:IdenRowsDetZero2}

\pause
\ExecuteMetaData[../det1.tex]{pf:IdenRowsDetZero3}  
\end{frame}
\begin{frame}
\ExecuteMetaData[../det1.tex]{pf:IdenRowsDetZero4}
\qed
\end{frame}








\begin{frame}{The $\nbyn{n}$ determinant is unique}
\lm[lm:DetFcnIsUnique]
\ExecuteMetaData[../det1.tex]{lm:DetFcnIsUnique}

\pause 
\pf 
\ExecuteMetaData[../det1.tex]{pf:DetFcnIsUnique}
\qed
\end{frame}



% ..... Four.I.3 .....
\section{The Permutation Expansion}
%..........
\begin{frame}{Multilinear}
\lm[lem:DetsMultilinear]
\ExecuteMetaData[../det1.tex]{lm:DetsMultilinear}

\pause
\pf
\ExecuteMetaData[../det1.tex]{pf:DetsMultilinear0}

\pause
\ExecuteMetaData[../det1.tex]{pf:DetsMultilinear1}
\end{frame}
\begin{frame}
\ExecuteMetaData[../det1.tex]{pf:DetsMultilinear2}
\end{frame}
\begin{frame}
\ExecuteMetaData[../det1.tex]{pf:DetsMultilinear3}
\qed
\end{frame}




\begin{frame}{Permutation matrices}
\df[df:permutation]
\ExecuteMetaData[../det1.tex]{df:permutation}

\pause
Recall Definition~Three.IV.\ref{df:PermutationMatrix},
that a \definend{permutation matrix}
is square, with all of its entries~$0$'s except for
a single~$1$ in each row and column.

\ExecuteMetaData[../det1.tex]{NotationForPermutationMatrices}
\end{frame}




\begin{frame}{Permutation expansion}
\df[df:PermutationExpansion]
\ExecuteMetaData[../det1.tex]{df:PermutationExpansion}

\pause
\medskip
\ExecuteMetaData[../det1.tex]{SummationForPermutationExpansion}
\end{frame}




\begin{frame}
\th[th:DetsExist]
\ExecuteMetaData[../det1.tex]{th:DetsExist}

\pause
\th[th:DeterminantOfAMatrixEqualsDeterminantOfTranspose]
\ExecuteMetaData[../det1.tex]{th:DeterminantOfAMatrixEqualsDeterminantOfTranspose}

\pause
\co[cor:ColSwapChgSign]
\ExecuteMetaData[../det1.tex]{co:ColSwapChgSign}
\pause
\pf
\ExecuteMetaData[../det1.tex]{pf:ColSwapChgSign}
\qed
\end{frame}




% ..... Four.I.4 .....
\section{Determinants Exist}
%..........
\begin{frame}{Inversion}
\df[df:Inversion]
\ExecuteMetaData[../det1.tex]{df:Inversion}
\end{frame}




%..........
\begin{frame}
\lm[le:SwapsChangeSgn]
\ExecuteMetaData[../det1.tex]{lm:SwapsChangeSgn}

\pause
\pf
\ExecuteMetaData[../det1.tex]{pf:SwapsChangeSgn0}
\end{frame}
\begin{frame}
\ExecuteMetaData[../det1.tex]{pf:SwapsChangeSgn1}
\end{frame}
\begin{frame}
\ExecuteMetaData[../det1.tex]{pf:SwapsChangeSgn2}
\qed
\end{frame}




%..........
\begin{frame}{Signum}
\df[df:Signum]
\ExecuteMetaData[../det1.tex]{df:Signum}
\end{frame}




%..........
\begin{frame}
\co[cor:ParityInversEqParitySwaps]
\ExecuteMetaData[../det1.tex]{co:ParityInversEqParitySwaps}

\pause
\pf
\ExecuteMetaData[../det1.tex]{pf:ParityInversEqParitySwaps}
\qed
\end{frame}




%..........
\begin{frame}{Determinants exist}
\ExecuteMetaData[../det1.tex]{DefiningDFunction}
\end{frame}
\begin{frame}
\lm[lm:DeterminantsExist]
\ExecuteMetaData[../det1.tex]{lm:DeterminantsExist}

\pf
\ExecuteMetaData[../det1.tex]{pf:DeterminantsExist0}

\pause
\ExecuteMetaData[../det1.tex]{pf:DeterminantsExist1}
% \end{frame}
% \begin{frame}

\pause
\ExecuteMetaData[../det1.tex]{pf:DeterminantsExist2}
\end{frame}
\begin{frame}
\ExecuteMetaData[../det1.tex]{pf:DeterminantsExist3}
\end{frame}
\begin{frame}
\ExecuteMetaData[../det1.tex]{pf:DeterminantsExist4}
\end{frame}
\begin{frame}
\ExecuteMetaData[../det1.tex]{pf:DeterminantsExist5}
\end{frame}
\begin{frame}
\ExecuteMetaData[../det1.tex]{pf:DeterminantsExist6}
\end{frame}
\begin{frame}
\ExecuteMetaData[../det1.tex]{pf:DeterminantsExist7}
\qed
\end{frame}




%..........
\begin{frame}{The determinant of the transpose}
\th[th:DetMatrixEqualsDetTrans]
\ExecuteMetaData[../det1.tex]{th:DetMatrixEqualsDetTrans}

\pf
\ExecuteMetaData[../det1.tex]{pf:DetMatrixEqualsDetTrans0}
\end{frame}
\begin{frame}
\ExecuteMetaData[../det1.tex]{pf:DetMatrixEqualsDetTrans1}
\qed
\end{frame}



% % ..... Four.I.2 .....
% \section{}
% %..........
% \begin{frame}
% \end{frame}




%...........................
% \begin{frame}
% \ExecuteMetaData[../gr3.tex]{GaussJordanReduction}
% \df[def:RedEchForm]
% 
% \end{frame}
\end{document}
