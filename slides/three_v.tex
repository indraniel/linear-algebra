% see: https://groups.google.com/forum/?fromgroups#!topic/comp.text.tex/s6z9Ult_zds
\makeatletter\let\ifGm@compatii\relax\makeatother 
\documentclass[10pt,t]{beamer}
\usefonttheme{professionalfonts}
\usefonttheme{serif}
\PassOptionsToPackage{pdfpagemode=FullScreen}{hyperref}
\PassOptionsToPackage{usenames,dvipsnames}{color}
% \DeclareGraphicsRule{*}{mps}{*}{}
\usepackage{../linalgjh}
\usepackage{present}
\usepackage{xr}\externaldocument{../map5} % read refs from .aux file
\usepackage{xr}\externaldocument{../map4} % 
\usepackage{catchfilebetweentags}
\usepackage{etoolbox} % from http://tex.stackexchange.com/questions/40699/input-only-part-of-a-file-using-catchfilebetweentags-package
\makeatletter
\patchcmd{\CatchFBT@Fin@l}{\endlinechar\m@ne}{}
  {}{\typeout{Unsuccessful patch!}}
\makeatother

\mode<presentation>
{
  \usetheme{boxes}
  \setbeamercovered{invisible}
  \setbeamertemplate{navigation symbols}{} 
}
\addheadbox{filler}{\ }  % create extra space at top of slide 
\hypersetup{colorlinks=true,linkcolor=blue} 

\title[Change of Basis] % (optional, use only with long paper titles)
{Three.V Change of Basis}

\author{\textit{Linear Algebra} \\ {\small Jim Hef{}feron}}
\institute{
  \texttt{http://joshua.smcvt.edu/linearalgebra}
}
\date{}


\subject{Change of Basis}
% This is only inserted into the PDF information catalog. Can be left
% out. 

\begin{document}
\begin{frame}
  \titlepage
\end{frame}

% =============================================
% \begin{frame}{Reduced Echelon Form} 
% \end{frame}



% ..... Three.V.1 .....
\section{Changing representations of vectors}
%..........
\begin{frame}{Coordinates vary with the basis}
\ex Consider this vector $\vec{v}\in\Re^3$ and bases for the space. 
\begin{equation*}
  \vec{v}=\colvec{1 \\ 2 \\ 3}
  \qquad
  \stdbasis_3=\sequence{\colvec{1 \\ 0 \\ 0}, \colvec{0 \\ 1 \\ 0}, \colvec{0 \\ 0 \\ 1}}
  \quad
  B=\sequence{\colvec{1 \\ 1 \\ 0}, \colvec{0 \\ 1 \\ 1}, \colvec{1 \\ 0 \\ 1}}
\end{equation*}
With respect to the different bases, the coordinates of $\vec{v}$ are different.
\begin{equation*}
  \rep{\vec{v}}{\stdbasis_3}=\colvec{1 \\ 2 \\ 3}
  \qquad
  \rep{\vec{v}}{B}=\colvec{0 \\ 2 \\ 1}
\end{equation*}
Here we will see how to convert between the two representations:
we will develop a formula that, given two bases for a space, 
converts the representation
of a vector with respect to the first basis to its representation with 
respect to the second.
\end{frame}


%..........
\begin{frame}{Change of basis matrix}
Think of translating from $\rep{\vec{v}}{B}$ to $\rep{\vec{v}}{D}$
as holding the vector constant. 
This is the arrow diagram.
\ExecuteMetaData[../map5.tex]{ChangeRepresentationOfVectorArrowDiagram}

\pause
\df[df:ChangeOfBasisMatrix]
\ExecuteMetaData[../map5.tex]{df:ChangeOfBasisMatrix}
\end{frame}



\begin{frame}
This result supports the definition.

\lm[le:ChBasisMatDoesChBases]
\ExecuteMetaData[../map5.tex]{lm:ChBasisMatDoesChBases}
\pf
\ExecuteMetaData[../map5.tex]{pf:ChBasisMatDoesChBases}
\qed
\end{frame}



\begin{frame}
\ex Two bases for $\polyspace_2$ are
$B=\sequence{1,1+x,1+x+x^2}$ and
$D=\sequence{x^2-1,x,x^2+1}$.
Compute $\rep{id}{B,D}$ in the same way that we compute the representation
of any function: find 
$\rep{\identity(1)}{D}$, $\rep{\identity(1+x)}{D}$, and
$\rep{\identity(1+x+x^2)}{D}$.
\begin{equation*}
  \rep{1}{D}=\colvec{-1/2 \\ 0 \\ 1/2}
  \quad
  \rep{1+x}{D}=\colvec{-1/2 \\ 1 \\ 1/2}
  \quad
  \rep{1+x+x^2}{D}=\colvec{0 \\ 1 \\ 1}
\end{equation*}
We put them together into the change of basis matrix.
\begin{equation*}
  \rep{id}{B,D}=
  \begin{mat}
    -1/2 &-1/2 &0 \\
     0   &1    &1 \\
    1/2  &1/2  &1 
  \end{mat}
\end{equation*}
For an example consider $\vec{v}=2-x+3x^2$.
\begin{equation*}
  \rep{\vec{v}}{B}=\colvec{3 \\ -4 \\ 3}
  \qquad
  \rep{\vec{v}}{D}
  =\colvec{1/2 \\ -1 \\ 5/2}
\end{equation*}
\end{frame}
\begin{frame}
The change of basis matrix does the conversion.
\begin{equation*}
  \begin{mat}
    -1/2 &-1/2 &0 \\
     0   &1    &1 \\
    1/2  &1/2  &1 
  \end{mat}
  \colvec{3 \\ -4 \\ 3}
  =
  \colvec{1/2 \\ -1 \\ 5/2}
\end{equation*}
\end{frame}



\begin{frame}
\lm[le:NonSingIsChBasis]
\ExecuteMetaData[../map5.tex]{lm:NonSingIsChBasis}
\pf
\ExecuteMetaData[../map5.tex]{pf:NonSingIsChBasis0}

\pause
\ExecuteMetaData[../map5.tex]{pf:NonSingIsChBasis1}
\end{frame}
\begin{frame}
\ExecuteMetaData[../map5.tex]{pf:NonSingIsChBasis2}
\begin{equation*}\hspace*{-1.5em}
  M_{i}(k)
    \colvec{
       c_1     \\
       \vdots  \\
       c_i    \\
       \vdots  \\
       c_n  }
  =
    \colvec{
       c_1     \\
       \vdots  \\
       kc_i    \\
       \vdots  \\
       c_n  }
  \quad
  P_{i,j}
    \colvec{
       c_1     \\
       \vdots  \\
       c_i    \\
       \vdots  \\
       c_j    \\
       \vdots  \\
       c_n  }
  =
    \colvec{
       c_1     \\
       \vdots  \\
       c_j    \\
       \vdots  \\
       c_i    \\
       \vdots  \\
       c_n  }
  \quad
  C_{i,j}(k)
    \colvec{
       c_1     \\
       \vdots  \\
       c_i    \\
       \vdots  \\
       c_j    \\
       \vdots  \\
       c_n  }
  =
    \colvec{
       c_1     \\
       \vdots  \\
       c_i    \\
       \vdots  \\
       kc_i+c_j    \\
       \vdots  \\
       c_n  }
\end{equation*}
\end{frame}
\begin{frame}
\ExecuteMetaData[../map5.tex]{pf:NonSingIsChBasis3}

\pause
\ExecuteMetaData[../map5.tex]{pf:NonSingIsChBasis4}
\end{frame}
\begin{frame}
\ExecuteMetaData[../map5.tex]{pf:NonSingIsChBasis5}
\qed
\end{frame}



\begin{frame}
\co[co:MatrixNonsingularIffChangesBasis]
\ExecuteMetaData[../map5.tex]{co:MatrixNonsingularIffChangesBasis}
\end{frame}




% ..... Three.V.2 .....
\section{Changing map representations}
%..........
\begin{frame}
The natural next step for us is to see how to convert $\rep{h}{B,D}$
to $\rep{h}{\hat{B},\hat{D}}$.
Here is the arrow diagram.
\ExecuteMetaData[../map5.tex]{ChangeRepresentationOfMapArrowDiagram}
\end{frame}




\begin{frame}
\ex
Consider the derivative map $\map{d/dx}{\polyspace_2}{\polyspace_2}$,
and consider also these two pairs of bases
$B=\sequence{1,1+x,1+x+x^2}, D=\sequence{1+x^2,x,1-x^2}$
and 
$\hat{B}=\sequence{1,x,x^2}, \hat{D}=\sequence{1+x,x+x^2,1+x^2}$.

We can find $H$ and~$\hat{H}$ using the methods we have already seen.
\begin{equation*}
  \rep{d/dx}{B,D}
  =
  \begin{mat}
    0 &1/2 &1/2 \\
    0 &0   &1 \\
    0 &1/2 &1/2
  \end{mat}
  \quad
  \rep{d/dx}{\hat{B},\hat{D}}
  =
  \begin{mat}
    0 &1/2  &1/2 \\
    0 &-1/2 &1/2 \\
    0 &1/2  &-1/2
  \end{mat}
\end{equation*}
To do the conversion we find these.
\begin{equation*}
  \rep{id}{\hat{B},B}=
  \begin{mat}
    1  &-1  &0 \\
    0  &1   &-1 \\
    0  &0   &1
  \end{mat}
  \quad
  \rep{id}{D,\hat{D}}=
  \begin{mat}
    0  &1/2    &1  \\
    0  &1/2    &-1 \\
    1  &-1/2   &0
  \end{mat}
\end{equation*}
Equation~($*$) says that this equals $\rep{d/dx}{\hat{B},\hat{D}}$.
\begin{equation*}
  \begin{mat}
    0  &1/2    &1  \\
    0  &1/2    &-1 \\
    1  &-1/2   &0
  \end{mat}
  \begin{mat}
    0 &1/2 &1/2 \\
    0 &0   &1 \\
    0 &1/2 &1/2
  \end{mat}
  \begin{mat}
    1  &-1  &0 \\
    0  &1   &-1 \\
    0  &0   &1
  \end{mat}  
\end{equation*}
\end{frame}




\begin{frame}
\df[def:MatEquiv]
\ExecuteMetaData[../map5.tex]{df:MatEquiv}
\pause
\co[le:MatEqIsSameMap]
\ExecuteMetaData[../map5.tex]{co:MatEqIsSameMap}

\pause
\medskip
\nearbyexercise{exer:MatEqIsEqRel} checks that
matrix equivalence is an equivalence relation.
Thus it  partitions\index{partition!matrix equivalence classes} 
the set of matrices into matrix equivalence classes.
\begin{center}
  \raisebox{.4in}{\begin{tabular}{l}
                    All matrices:
                  \end{tabular}}
  \includegraphics{../ch3.23}
  \raisebox{.4in}{\begin{tabular}{l}
                     $H$ matrix equivalent \\ to $\hat{H}$
                  \end{tabular}}
\end{center}
\end{frame}




\begin{frame}{Canonical form for matrix equivalence}
\th[th:CanonFormForMatEquiv]
\ExecuteMetaData[../map5.tex]{th:CanonFormForMatEquiv}

\pause
\medskip
\ExecuteMetaData[../map5.tex]{BlockPartialIdentityForm}
\end{frame}
\begin{frame}
\pf
\ExecuteMetaData[../map5.tex]{th:CanonFormForMatEquiv}
\qed
\end{frame}




\begin{frame}{Matrix equivalence is characterized by rank}
\co[co:MatrixEquivalentIffSameRank]
\ExecuteMetaData[../map5.tex]{co:MatrixEquivalentIffSameRank}
\pause
\pf
\ExecuteMetaData[../map5.tex]{pf:MatrixEquivalentIffSameRank}
\qed

\pause
\ex
These two matrices are not matrix equivalent 
because Gauss's Method shows
that the first has rank~$3$ while the second has rank~$2$.
\begin{equation*}
  \begin{mat}[r]
    2  &3  &0 &-1 \\
    2  &2  &1 &1  \\
    3  &1  &0 &3
  \end{mat}
  \quad
  \begin{mat}[r]
    1  &5  &1 &4 \\
    2  &0  &5 &1  \\
    3  &-5 &9 &-2
  \end{mat}
\end{equation*}
\ex
These two are matrix equivalent.
\begin{equation*}
  \begin{mat}[r]
    1  &1  &1  &1  &1  \\
   -1  &-1 &-1 &-1 &-1 \\
    2  &0  &0  &3  &0 
  \end{mat}
  \quad
  \begin{mat}
    1  &0  &0  &0  &0  \\
    0  &1  &0  &0  &0  \\
    0  &0  &0  &0  &0
  \end{mat}
\end{equation*}
\end{frame}




%...........................
% \begin{frame}
% \ExecuteMetaData[../gr3.tex]{GaussJordanReduction}
% \df[def:RedEchForm]
% 
% \end{frame}
\end{document}
