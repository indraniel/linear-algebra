% see: https://groups.google.com/forum/?fromgroups#!topic/comp.text.tex/s6z9Ult_zds
\makeatletter\let\ifGm@compote\relax\makeatother 
\documentclass[10pt,t,serif,professionalfont]{beamer}
\PassOptionsToPackage{pdfpagemode=FullScreen}{hyperref}
\PassOptionsToPackage{usenames,dvipsnames}{color}
% \DeclareGraphicsRule{*}{mps}{*}{}
\usepackage{../linalgjh}
\usepackage{present}
\usepackage{xr}\externaldocument{../det2} % read refs from .aux file
\usepackage{catchfilebetweentags}
\usepackage{etoolbox} % from http://tex.stackexchange.com/questions/40699/input-only-part-of-a-file-using-catchfilebetweentags-package
\makeatletter
\patchcmd{\CatchFBT@Fin@l}{\endlinechar\m@ne}{}
  {}{\typeout{Unsuccessful patch!}}
\makeatother

\mode<presentation>
{
  \usetheme{boxes}
  \setbeamercovered{invisible}
  \setbeamertemplate{navigation symbols}{} 
}
\addheadbox{filler}{\ }  % create extra space at top of slide 
\hypersetup{colorlinks=true,linkcolor=blue} 

\title[Geometry of Determinants] % (optional, use only with long paper titles)
{Four.II Geometry of Determinants}

\author{\textit{Linear Algebra} \\ {\small Jim Hef{}feron}}
\institute{
  \texttt{http://joshua.smcvt.edu/linearalgebra}
}
\date{}


\subject{Geometry of Determinants}
% This is only inserted into the PDF information catalog. Can be left
% out. 

\begin{document}
\begin{frame}
  \titlepage
\end{frame}

% =============================================
% \begin{frame}{Reduced Echelon Form} 
% \end{frame}



% ..... Four.II .....
\section{Determinants as size functions}
%..........
\begin{frame}{Box}
This parallelogram is defined by the two vectors.
\centergraphic{../ch4.30}

\df[df:Box]
\ExecuteMetaData[../det2.tex]{df:Box}

\medskip
The box is like the span, except that the scalars are limited to the unit 
interval. 
\end{frame}
\begin{frame}{Area of a two dimensional box}
\centergraphic{../ch4.31}
\begin{align*}
  \text{box area}
  &=\text{rectangle area}-\text{area of $A$}-\cdots-\text{area of F} \\
  &=(x_1+x_2)(y_1+y_2)-x_2y_1-x_1y_1/2        \\
    &\quad-x_2y_2/2-x_2y_2/2-x_1y_1/2-x_2y_1         \\
  &=x_1y_2-x_2y_1        
\end{align*}
\pause
That area equals the value of the determinant.
\begin{equation*}
  \begin{vmat}
    x_1  &x_2  \\
    y_1  &y_2
  \end{vmat}
  =x_1y_2-x_2y_1
\end{equation*}
\end{frame}
\begin{frame}
We will argue here that
the properties of determinants 
make good postulates for a function 
that gives the size of the box formed by the columns of the matrix.  

\pause
In line with the above use of column vectors,
in this section we consider the determinant as a function of the columns
of the matrix.
Note that because the determinant of a matrix equals 
the determinant of its transpose, we can change the determinant properties
from statements about rows to statements about columns.
For instance, the first property as given earlier says that a
determinant is unaffected by a row operation $k\rho_i+\rho_j$ (with $i\neq j$).
By transposing we have the property that a determinant
does not change under combinations of columns. 
\end{frame}




\begin{frame}{Definition of determinant reinterpreted}
Recall property~(3) from the definition of determinant,
that rescaling a column rescales the entire determinant
$\det(\ldots,k\vec{v}_i,\ldots)=k\det(\ldots,\vec{v}_i,\ldots)$.
This fits with the idea that the determinant
gives the size of the box formed by the columns of the matrix:
if we scale a column by a factor~$k$ then the size of the box
scales by that factor. 
\begin{center}
  \includegraphics{../ch4.32}
  \qquad
  \includegraphics{../ch4.33}
\end{center}
% \end{frame}
% \begin{frame}

\pause
Property~(1) says that the determinant is unaffected by 
combining columns.
The picture 
\begin{center}
  \includegraphics{../ch4.34}
  \quad
  \includegraphics{../ch4.35}
\end{center}   
shows that the box
formed by $\vec{v}$ and~$k\vec{v}+\vec{w}$ 
is more slanted than the original box but the two have
the same base and height, and hence the same area.
\end{frame}
\begin{frame}
As we noted after the definition, property~(2) is a consequence of the 
other properties so we leave it aside for the moment.  

Property~(4) says that the determinant of the identity matrix is~$1$.
\centergraphic{../ch4.36}
\end{frame}




\begin{frame}{Orientation}
\re[re:PropertyTwoGivesSign] 
Although property~(2) is redundant it says something notable.
Consider these two.
\begin{center} \small
  \begin{tabular}{c@{\hspace*{8em}}c}
    \includegraphics{../ch4.37}  
      &\includegraphics{../ch4.38}  \\[.25ex]
    \ $\begin{vmat}[r]
        4  &1   \\
        2  &3
      \end{vmat}=10$
      &\ $\begin{vmat}[r]
          1  &4   \\
          3  &2
        \end{vmat}=-10$
  \end{tabular}
\end{center}
Swapping changes the sign;
on the left we take $\vec{u}$ first in the matrix and then follow the
counterclockwise arc to $\vec{v}$,
following the counterclockwise arc and
and get a positive size, while
on the right following the clockwise arc gives a negative size.
The sign returned by the size function, the determinant, reflects the 
\definend{orientation}\index{box!orientation}\index{orientation} 
or \definend{sense}\index{box!sense}\index{sense} of the box.
\pause
We see the same thing if we use Property~(3) with a negative scalar.
\end{frame}
\begin{frame}{Aside about orientation}
Suppose that in three space 
starting with these two vectors we want to form a box 
with positive size.
\begin{equation*}
  \vec{v}_1=\colvec{1 \\ 4 \\ 1},\,
  \vec{v}_2=\colvec{-2 \\ 3 \\ 1}
  %\vec{v}_3=\colvec{0 \\ -1 \\ 2}
  \qquad
  \vcenteredhbox{\includegraphics{asy/four_ii_orientation.png}}
\end{equation*}
Those two vectors form a plane, which divides three-space in two halves.
The $\vec{v}_3$ shown is on the side of the plane containing vectors
with this property: if a person at the tip looks down at the box and traces
out the parallelogram from $\zero$, to $\vec{v}_1$, to $\vec{v}_1+\vec{v}_2$,
to $\vec{v}_2$, and back to $\zero$ then their trace looks to them
to be counterclockwise.
Any such vector will make a positive-sized box.
\end{frame}
\begin{frame}
The vector shown is this.
\begin{equation*}
  \vec{v}_3=\colvec{0 \\ -1 \\ 2}
  \qquad
  \begin{vmat}
    1 &-2 &0 \\
    4 &3 &-1 \\
    1 &1 &2
  \end{vmat}=25
\end{equation*}
\pause
Any vector on the other side of the plane,
such as $-\vec{v}_3$, will have the same trace look clockwise and will
give a negative determinant.
\begin{equation*}
  \begin{vmat}
    1 &-2 &0 \\
    4 &3 &1 \\
    1 &1 &-2
  \end{vmat}=-25
\end{equation*}
\end{frame}





%...........................
% \begin{frame}g
% \ExecuteMetaData[../gr3.tex]{GaussJordanReduction}
% \df[def:RedEchForm]
% 
% \end{frame}
\end{document}
