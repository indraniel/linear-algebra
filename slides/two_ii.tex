% see: https://groups.google.com/forum/?fromgroups#!topic/comp.text.tex/s6z9Ult_zds
\makeatletter\let\ifGm@compatii\relax\makeatother 
\documentclass[10pt,t,serif,professionalfont]{beamer}
\PassOptionsToPackage{pdfpagemode=FullScreen}{hyperref}
\PassOptionsToPackage{usenames,dvipsnames}{color}
% \DeclareGraphicsRule{*}{mps}{*}{}
\usepackage{../linalgjh}
\usepackage{present}
\usepackage{xr}\externaldocument{../vs2} % read refs from .aux file
\usepackage{catchfilebetweentags}
\usepackage{etoolbox} % from http://tex.stackexchange.com/questions/40699/input-only-part-of-a-file-using-catchfilebetweentags-package
\makeatletter
\patchcmd{\CatchFBT@Fin@l}{\endlinechar\m@ne}{}
  {}{\typeout{Unsuccessful patch!}}
\makeatother

\mode<presentation>
{
  \usetheme{boxes}
  \setbeamercovered{invisible}
  \setbeamertemplate{navigation symbols}{} 
}
\addheadbox{filler}{\ }  % create extra space at top of slide 
\hypersetup{colorlinks=true,linkcolor=blue} 

\title[Linear Independence] % (optional, use only with long paper titles)
{Two.II Linear Independence}

\author{\textit{Linear Algebra} \\ {\small Jim Hef{}feron}}
\institute{
  \texttt{http://joshua.smcvt.edu/linearalgebra}
}
\date{}


\subject{Linear Independence}
% This is only inserted into the PDF information catalog. Can be left
% out. 

\begin{document}
\begin{frame}
  \titlepage
\end{frame}

% =============================================
% \begin{frame}{Reduced Echelon Form} 
% \end{frame}



% ..... Two.I.1 .....
\section{Definition and examples}
%..........
\begin{frame}{Linear independence}
\df[def:LinInd]
\ExecuteMetaData[../vs2.tex]{df:LinInd}

\pause\medskip
\ExecuteMetaData[../vs2.tex]{LinInd}
\end{frame}



%..........
\begin{frame}
\lm[le:LDIffANonTrivLinRel]
\ExecuteMetaData[../vs2.tex]{lm:LDIffANonTrivLinRel}

\pause
\pf
\ExecuteMetaData[../vs2.tex]{pf:LDIffANonTrivLinRel}
\qed
\end{frame}



%..........
\begin{frame}
\th[th:AlwaysAnLDSubset]
\ExecuteMetaData[../vs2.tex]{th:AlwaysAnLDSubset}

\pause
\pf
\ExecuteMetaData[../vs2.tex]{pf:AlwaysAnLDSubset0}

\pause
\ExecuteMetaData[../vs2.tex]{pf:AlwaysAnLDSubset1}

\pause
\ExecuteMetaData[../vs2.tex]{pf:AlwaysAnLDSubset2}
\qed
\end{frame}



%..........
\begin{frame}
\lm[le:SubsetPreserveLI]
\ExecuteMetaData[../vs2.tex]{lm:SubsetPreserveLI}

\pause
\pf
\ExecuteMetaData[../vs2.tex]{pf:SubsetPreserveLI}
\qed

\pause
\ExecuteMetaData[../vs2.tex]{SubsetPreserveLI}
\end{frame}



%..................
\begin{frame}
\lm[le:VecInSpanIffSpanUnchByAddVec]
\ExecuteMetaData[../vs2.tex]{lm:VecInSpanIffSpanUnchByAddVec}

\pause
\pf
\ExecuteMetaData[../vs2.tex]{pf:VecInSpanIffSpanUnchByAddVec0}

\pause
\ExecuteMetaData[../vs2.tex]{pf:VecInSpanIffSpanUnchByAddVec1}
\qed
\end{frame}



%..................
\begin{frame}
\lm[le:SUnionXiLIIffXiNotInSpan]
\ExecuteMetaData[../vs2.tex]{lm:SUnionXiLIIffXiNotInSpan}

\pause
\pf
\ExecuteMetaData[../vs2.tex]{pf:SUnionXiLIIffXiNotInSpan0}

\pause
\ExecuteMetaData[../vs2.tex]{pf:SUnionXiLIIffXiNotInSpan1}
\qed
\end{frame}



%..................
\begin{frame}
\co[cor:LDMeansLC]
\ExecuteMetaData[../vs2.tex]{co:LDMeansLC}

\pause
\pf
\ExecuteMetaData[../vs2.tex]{pf:LDMeansLC}
\qed
\end{frame}



%..................
\begin{frame}{Linear independence and subset}
\ExecuteMetaData[../vs2.tex]{IndependenceAndSubsetTable}
\end{frame}



%...........................
% \begin{frame}
% \ExecuteMetaData[../gr3.tex]{GaussJordanReduction}
% \df[def:RedEchForm]
% 
% \end{frame}
\end{document}
