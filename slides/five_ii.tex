% see: https://groups.google.com/forum/?fromgroups#!topic/comp.text.tex/s6z9Ult_zds
\makeatletter\let\ifGm@compote\relax\makeatother 
\documentclass[10pt,t,serif,professionalfont]{beamer}
\PassOptionsToPackage{pdfpagemode=FullScreen}{hyperref}
\PassOptionsToPackage{usenames,dvipsnames}{color}
% \DeclareGraphicsRule{*}{mps}{*}{}
\usepackage{../linalgjh}
\usepackage{present}
\usepackage{xr}\externaldocument{../jc2} % read refs from .aux file
\usepackage{catchfilebetweentags}
\usepackage{etoolbox} % from http://tex.stackexchange.com/questions/40699/input-only-part-of-a-file-using-catchfilebetweentags-package
\makeatletter
\patchcmd{\CatchFBT@Fin@l}{\endlinechar\m@ne}{}
  {}{\typeout{Unsuccessful patch!}}
\makeatother

\usepackage{polynom}  % for polynomial long division

\mode<presentation>
{
  \usetheme{boxes}
  \setbeamercovered{invisible}
  \setbeamertemplate{navigation symbols}{} 
}
\addheadbox{filler}{\ }  % create extra space at top of slide 
\hypersetup{colorlinks=true,linkcolor=blue} 

\title[Similarity] % (optional, use only with long paper titles)
{Five.II Similarity}

\author{\textit{Linear Algebra} \\ {\small Jim Hef{}feron}}
\institute{
  \texttt{http://joshua.smcvt.edu/linearalgebra}
}
\date{}


\subject{Similarity}
% This is only inserted into the PDF information catalog. Can be left
% out. 

\begin{document}
\begin{frame}
  \titlepage
\end{frame}

% =============================================
\begin{frame}
\vspace*{-2ex}
\ExecuteMetaData[../jc2.tex]{SimilarityMotiviation0}  
\pause  
\ExecuteMetaData[../jc2.tex]{SimilarityMotiviation1}  
\end{frame}




% ..... Five.II.1 .....
\section{Definition and Examples}
\begin{frame}{Similar matrices}
\df[df:Similar]
\ExecuteMetaData[../jc2.tex]{df:Similar}  
\end{frame}




% ..... Five.II.2 .....
\section{Diagonalizability}
\begin{frame}{Diagonalizable matrix}
\df[df:Diagonalizable]
\ExecuteMetaData[../jc2.tex]{df:Diagonalizable}  

\pause
\pf
\ExecuteMetaData[../jc2.tex]{df:Diagonalizable}  
\qed
\end{frame}




\begin{frame}
\lm[lm:DiagIffBasisOfEigens]
\ExecuteMetaData[../jc2.tex]{lm:DiagIffBasisOfEigens}  

\pause
\pf
\ExecuteMetaData[../jc2.tex]{pf:DiagIffBasisOfEigens}  
\qed
\end{frame}




% ..... Five.II.3 .....
\section{Eigenvalues and Eigenvectors}
\begin{frame}{Eigenvalues and eigenvectors}
\df[def:Eigen]
\ExecuteMetaData[../jc2.tex]{df:Eigen}  

\pause
\df[df:EigenOfMatrix]
\ExecuteMetaData[../jc2.tex]{df:EigenOfMatrix}  
\end{frame}




\begin{frame}{Characteristic polynomial}
\df[df:CharacteristicPoly]
\ExecuteMetaData[../jc2.tex]{df:CharacteristicPoly}

\pause
\lm[le:MapNonTrivSpHasEigen] 
\ExecuteMetaData[../jc2.tex]{lm:MapNonTrivSpHasEigen}

\pause
\pf
\ExecuteMetaData[../jc2.tex]{pf:MapNonTrivSpHasEigen}
\qed
\end{frame}




\begin{frame}{Eigenspace}
\df[df:Eigenspace]
\ExecuteMetaData[../jc2.tex]{df:Eigenspace}

\pause
\lm[le:EigSpaceIsSubSp] 
\ExecuteMetaData[../jc2.tex]{lm:EigSpaceIsSubSp}

\pause
\pf
\ExecuteMetaData[../jc2.tex]{pf:EigSpaceIsSubSp}
\qed
\end{frame}




\begin{frame}
\th[th:DistEValueGivesLIEvecs]
\ExecuteMetaData[../jc2.tex]{th:DistEValueGivesLIEvecs}

\pause
\pf
\ExecuteMetaData[../jc2.tex]{pf:DistEValueGivesLIEvecs0}
\end{frame}
\begin{frame}
\ExecuteMetaData[../jc2.tex]{pf:DistEValueGivesLIEvecs1}
\qed
\end{frame}




%...........................
% \begin{frame}g
% \ExecuteMetaData[../gr3.tex]{GaussJordanReduction}
% \df[def:RedEchForm]
% 
% \end{frame}
\end{document}
