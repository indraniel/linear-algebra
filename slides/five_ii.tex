% see: https://groups.google.com/forum/?fromgroups#!topic/comp.text.tex/s6z9Ult_zds
\makeatletter\let\ifGm@compote\relax\makeatother 
\documentclass[10pt,t,serif,professionalfont]{beamer}
\PassOptionsToPackage{pdfpagemode=FullScreen}{hyperref}
\PassOptionsToPackage{usenames,dvipsnames}{color}
% \DeclareGraphicsRule{*}{mps}{*}{}
\usepackage{../linalgjh}
\usepackage{present}
\usepackage{xr}\externaldocument{../jc2} % read refs from .aux file
\usepackage{catchfilebetweentags}
\usepackage{etoolbox} % from http://tex.stackexchange.com/questions/40699/input-only-part-of-a-file-using-catchfilebetweentags-package
\makeatletter
\patchcmd{\CatchFBT@Fin@l}{\endlinechar\m@ne}{}
  {}{\typeout{Unsuccessful patch!}}
\makeatother

\usepackage{polynom}  % for polynomial long division

\mode<presentation>
{
  \usetheme{boxes}
  \setbeamercovered{invisible}
  \setbeamertemplate{navigation symbols}{} 
}
\addheadbox{filler}{\ }  % create extra space at top of slide 
\hypersetup{colorlinks=true,linkcolor=blue} 

\title[Similarity] % (optional, use only with long paper titles)
{Five.II Similarity}

\author{\textit{Linear Algebra} \\ {\small Jim Hef{}feron}}
\institute{
  \texttt{http://joshua.smcvt.edu/linearalgebra}
}
\date{}


\subject{Similarity}
% This is only inserted into the PDF information catalog. Can be left
% out. 

\begin{document}
\begin{frame}
  \titlepage
\end{frame}

% =============================================
\begin{frame}
\vspace*{-2ex}
\ExecuteMetaData[../jc2.tex]{SimilarityMotiviation0}  
\pause  
\ExecuteMetaData[../jc2.tex]{SimilarityMotiviation1}  
\end{frame}




% ..... Five.II.1 .....
\section{Definition and Examples}
\begin{frame}{Similar matrices}
\df[df:Similar]
\ExecuteMetaData[../jc2.tex]{df:Similar}  

\ex
Consider the derivative map $\map{d/dx}{\polyspace_2}{\polyspace_2}$.
Fix the basis $B=\sequence{1,x,x^2}$ 
and the basis $D=\sequence{1,1+x,1+x+x^2}$.
In this arrow diagram we will first get $S$, and then calculate $T$ from it.
\begin{equation*}
  \begin{CD}
    V_{\wrt{B}}                   @>t>S>        V_{\wrt{B}}       \\
    @V{\scriptstyle\identity} VV              @V{\scriptstyle\identity} VV \\
    V_{\wrt{D}}                   @>t>T>        V_{\wrt{D}}
  \end{CD}
\end{equation*}
\pause
The action of $d/dx$ on the 
elements of the basis $B$ is $1\mapsto 0$, $x\mapsto 1$, and $x^2\mapsto 2x$.
\begin{equation*}
  \rep{d/dx(1)}{B}=\colvec{0 \\ 0 \\ 0}
  \quad
  \rep{d/dx(x)}{B}=\colvec{1 \\ 0 \\ 0}
  \quad
  \rep{d/dx(x^2)}{B}=\colvec{0 \\ 2 \\ 0}
\end{equation*}
\end{frame}
\begin{frame}
So we have this matrix representation of the map.
\begin{equation*}
  S=\rep{d/dx}{B,B}=
  \begin{mat}
    0 &1 &0 \\
    0 &0 &2 \\
    0 &0 &0
  \end{mat}
\end{equation*}
\pause
Recall that the matrix changing bases from $B$ to $D$ is $\rep{\identity}{B,D}$.
We find these by eye
\begin{equation*}
  \rep{1}{D}=\colvec{1 \\ 0 \\ 0}
  \quad
  \rep{x}{D}=\colvec{-1 \\ 1 \\ 0}
  \quad
  \rep{x^2}{D}=\colvec{0 \\ -1 \\ 1}
\end{equation*}
to get this.
\begin{equation*}
  P=
  \begin{mat}
    1 &-1 &0  \\
    0 &1  &-1 \\
    0 &0  &1
  \end{mat}
  \qquad
  P^{-1}=
  \begin{mat}
    1 &1  &1  \\
    0 &1  &1 \\
    0 &0  &1
  \end{mat}
\end{equation*}
Now, by following the arrow diagram we have $T=PSP^{-1}$.
\begin{equation*}
  T=
  \begin{mat}
    0 &1 &-1 \\
    0 &0 &2  \\
    0 &0 &0
  \end{mat}
\end{equation*}
\end{frame}
\begin{frame}
We underline what the arrow diagram says 
\begin{equation*}
  \begin{CD}
    V_{\wrt{B}}                   @>t>S>        V_{\wrt{B}}       \\
    @V{\scriptstyle\identity} VV              @V{\scriptstyle\identity} VV \\
    V_{\wrt{D}}                   @>t>T>        V_{\wrt{D}}
  \end{CD}
\end{equation*}
by calculating $T$ directly.
The effect of the map on the basis elements is 
$d/dx(1)=0$, $d/dx(1+x)=1$, and $d/dx(1+x+x^2)=1+2x$.
Representating of those with respect to $D$
\begin{equation*}
  \rep{0}{D}=\colvec{0 \\ 0 \\ 0}
  \quad
  \rep{1}{D}=\colvec{1 \\ 0 \\ 0}
  \quad
  \rep{1+2x}{D}=\colvec{-1 \\ 2 \\ 0}
\end{equation*}
gives the same matrix $\rep{d/dx}{D,D}$ as we found above.
\end{frame}
\begin{frame}
We don't need to consider the underlying maps.
We can just multiply matrices.  

\ex
Where 
\begin{equation*}
  S=
  \begin{mat}
    0 &-1 &-2 \\
    2 &3 &2   \\
    4 &5 &2
  \end{mat}
  \qquad
  P=
  \begin{mat}
    1 &1 &0 \\
   -1 &1 &0   \\
    0 &0 &3
  \end{mat}
\end{equation*}
(note that $P$ is nonsingular) we can compute this $T=PSP^{-1}$.
\begin{equation*}
  T=
  \begin{mat}
    2   &0   &0 \\
    3   &1   &4/3 \\
   27/2 &3/2 &2
  \end{mat}
\end{equation*}

\pause
\ex[ex:OnlyZeroSimToZero]
\ExecuteMetaData[../jc2.tex]{ex:OnlyZeroSimToZero}  
\end{frame}



\begin{frame}{Similarity is an equivalence}
\nearbyexercise{exer:SimIsEquivRel} checks that
similarity is an equivalence relation.

\pause
Since matrix similarity is a special case of matrix equivalence, 
if two matrices are similar then they are matrix equivalent.
What about the converse:~must any two matrix equivalent square matrices be 
similar?
No; the matrix equivalence class
of an identity consists of all nonsingular matrices of that size. 

\pause
So some matrix equivalence classes
split into two or more similarity classes\Dash similarity gives a finer
partition than does equivalence.
This pictures some matrix equivalence classes subdivided into
similarity classes.
\centergraphic{../ch5.4} 
\end{frame}




% ..... Five.II.2 .....
\section{Diagonalizability}
\begin{frame}
\df[df:Diagonalizable]
\ExecuteMetaData[../jc2.tex]{df:Diagonalizable}  

\ex
This matrix
\begin{equation*}
  \begin{mat}
    6 &-1  &-1 \\
    2 &11  &-1 \\
   -6 &-5  &7
  \end{mat}
\end{equation*}
is diagonalizable by using this
\begin{equation*}
  P=
  \begin{mat}
    1/2 &1/4  &1/4 \\
   -1/2 &1/4  &1/4 \\
   -1/2 &-3/4 &1/4
  \end{mat}
  \quad
  P^{-1}
  \begin{mat}
    1 &-1 &0 \\
    0 &1 &-1 \\
    2 &1 &1
  \end{mat}
\end{equation*}
to get this $T=PSP^{-1}$.
\begin{equation*}
  T=
  \begin{mat}
    4 &0 &0 \\
    0 &8 &0 \\
    0 &0 &12
  \end{mat}
\end{equation*}
\end{frame}
\begin{frame}
\ex  
\ExecuteMetaData[../jc2.tex]{ex:NotDiagonalizable}  
\end{frame}




\begin{frame}
\lm[lm:DiagIffBasisOfEigens]
\ExecuteMetaData[../jc2.tex]{lm:DiagIffBasisOfEigens}  

\pause
\pf
\ExecuteMetaData[../jc2.tex]{pf:DiagIffBasisOfEigens}  
\qed
\end{frame}




% ..... Five.II.3 .....
\section{Eigenvalues and Eigenvectors}
\begin{frame}{Eigenvalues and eigenvectors}
\df[def:Eigen]
\ExecuteMetaData[../jc2.tex]{df:Eigen}  

\pause
\df[df:EigenOfMatrix]
\ExecuteMetaData[../jc2.tex]{df:EigenOfMatrix}  
\end{frame}




\begin{frame}{Characteristic polynomial}
\df[df:CharacteristicPoly]
\ExecuteMetaData[../jc2.tex]{df:CharacteristicPoly}

\pause
\lm[le:MapNonTrivSpHasEigen] 
\ExecuteMetaData[../jc2.tex]{lm:MapNonTrivSpHasEigen}

\pause
\pf
\ExecuteMetaData[../jc2.tex]{pf:MapNonTrivSpHasEigen}
\qed
\end{frame}




\begin{frame}{Eigenspace}
\df[df:Eigenspace]
\ExecuteMetaData[../jc2.tex]{df:Eigenspace}

\pause
\lm[le:EigSpaceIsSubSp] 
\ExecuteMetaData[../jc2.tex]{lm:EigSpaceIsSubSp}

\pause
\pf
\ExecuteMetaData[../jc2.tex]{pf:EigSpaceIsSubSp}
\qed
\end{frame}




\begin{frame}
\th[th:DistEValueGivesLIEvecs]
\ExecuteMetaData[../jc2.tex]{th:DistEValueGivesLIEvecs}

\pause
\pf
\ExecuteMetaData[../jc2.tex]{pf:DistEValueGivesLIEvecs0}
\end{frame}
\begin{frame}
\ExecuteMetaData[../jc2.tex]{pf:DistEValueGivesLIEvecs1}
\qed
\end{frame}




%...........................
% \begin{frame}g
% \ExecuteMetaData[../gr3.tex]{GaussJordanReduction}
% \df[def:RedEchForm]
% 
% \end{frame}
\end{document}
