% see: https://groups.google.com/forum/?fromgroups#!topic/comp.text.tex/s6z9Ult_zds
\makeatletter\let\ifGm@compatii\relax\makeatother 
\documentclass[10pt,t,serif,professionalfont]{beamer}
\PassOptionsToPackage{pdfpagemode=FullScreen}{hyperref}
\PassOptionsToPackage{usenames,dvipsnames}{color}
% \DeclareGraphicsRule{*}{mps}{*}{}
\usepackage{../linalgjh}
\usepackage{present}
\usepackage{xr}\externaldocument{../map4} % read refs from .aux file
\usepackage{catchfilebetweentags}
\usepackage{etoolbox} % from http://tex.stackexchange.com/questions/40699/input-only-part-of-a-file-using-catchfilebetweentags-package
\makeatletter
\patchcmd{\CatchFBT@Fin@l}{\endlinechar\m@ne}{}
  {}{\typeout{Unsuccessful patch!}}
\makeatother

\mode<presentation>
{
  \usetheme{boxes}
  \setbeamercovered{invisible}
  \setbeamertemplate{navigation symbols}{} 
}
\addheadbox{filler}{\ }  % create extra space at top of slide 
\hypersetup{colorlinks=true,linkcolor=blue} 

\title[Matrix Operations] % (optional, use only with long paper titles)
{Three.IV Matrix Operations}

\author{\textit{Linear Algebra} \\ {\small Jim Hef{}feron}}
\institute{
  \texttt{http://joshua.smcvt.edu/linearalgebra}
}
\date{}


\subject{Matrix Operations}
% This is only inserted into the PDF information catalog. Can be left
% out. 

\begin{document}
\begin{frame}
  \titlepage
\end{frame}

% =============================================
% \begin{frame}{Reduced Echelon Form} 
% \end{frame}



% ..... Three.IV.1 .....
\section{Sums and Scalar Products}
%..........
\begin{frame}{Definition of matrix sum and scalar multiple}
\df[def:SumScalarProdMats]
\ExecuteMetaData[../map4.tex]{df:SumScalarProdMats}
\end{frame}




%..........
\begin{frame}
\th[th:MatOpsRepMapOps]
\ExecuteMetaData[../map4.tex]{th:MatOpsRepMapOps}
\end{frame}




%..........
\begin{frame}
\df[df:ZeroMatrix]
\ExecuteMetaData[../map4.tex]{df:ZeroMatrix}
\end{frame}



% ..... Three.IV.2 .....
\section{Matrix Multiplication}
%..........
\begin{frame}
\lm[lm:CompositionOfLinearMapsIsLinear]
\ExecuteMetaData[../map4.tex]{lm:CompositionOfLinearMapsIsLinear}

\pause
\pf
\ExecuteMetaData[../map4.tex]{pf:CompositionOfLinearMapsIsLinear}
\qed
\end{frame}




%..........
\begin{frame}{Definition of matrix multiplication}
\df[def:MatMult]
\ExecuteMetaData[../map4.tex]{df:MatMult}
\end{frame}




%..........
\begin{frame}{Matrix multiplication represents composition}
\th[th:MatMultRepComp]
\ExecuteMetaData[../map4.tex]{th:MatMultRepComp}
\pause
\pf
\ExecuteMetaData[../map4.tex]{th:MatMultRepComp}
\qed
\end{frame}




%..........
\begin{frame}{Arrow diagrams}
% \ExecuteMetaData[../map4.tex]{MatMultArrowDiag0}
This pictures the relationship between maps and matrices.
\centergraphic{../ch3.20}
\pause
\ExecuteMetaData[../map4.tex]{MatMultArrowDiag1}
\end{frame}




%..........
\begin{frame}
\th[th:MatMultWellBehaved]
\ExecuteMetaData[../map4.tex]{th:MatMultWellBehaved}
\pause
\pf
\ExecuteMetaData[../map4.tex]{pf:MatMultWellBehaved0}

\pause
\ExecuteMetaData[../map4.tex]{pf:MatMultWellBehaved1}
\qed
\end{frame}




%..........
\begin{frame}
\df[df:ZeroMatrix]
\ExecuteMetaData[../map4.tex]{df:ZeroMatrix}
\end{frame}



% ..... Three.IV.2 .....
\section{Mechanics of Matrix Multiplication}
%..........
\begin{frame}
\df[df:UnitMatrix]
\ExecuteMetaData[../map4.tex]{df:UnitMatrix}
\end{frame}




%..........
\begin{frame}
\lm[lm:ColsAndRowsInMatrixMult]
\ExecuteMetaData[../map4.tex]{lm:ColsAndRowsInMatrixMult}
\end{frame}
\begin{frame}
\pf
\ExecuteMetaData[../map4.tex]{pf:ColsAndRowsInMatrixMult}
\qed
\end{frame}




%..........
\begin{frame}
\df[df:MainDiagonal]
\ExecuteMetaData[../map4.tex]{df:MainDiagonal}
\pause
\df[df:IdentityMatrix]
\ExecuteMetaData[../map4.tex]{df:IdentityMatrix}
\end{frame}




%..........
\begin{frame}
\df[df:DiagonalMatrix]
\ExecuteMetaData[../map4.tex]{df:DiagonalMatrix}
\end{frame}




%..........
\begin{frame}
\df[df:PermutationMatrix]
\ExecuteMetaData[../map4.tex]{df:PermutationMatrix}
\end{frame}




%..........
\begin{frame}
\df[df:ElementaryReductionMatrices]
\ExecuteMetaData[../map4.tex]{df:ElementaryReductionMatrices}
\end{frame}




\begin{frame}
\lm[GrByMatMult]
\ExecuteMetaData[../map4.tex]{lm:GrByMatMult}
\pf
Clear.
\qed
\co[cor:ReducViaMatrices]
\ExecuteMetaData[../map4.tex]{co:ReducViaMatrices}
\end{frame}



% ..... Three.IV.3 .....
\section{Inverses}
%..........
\begin{frame}{Definition of matrix inverse}
\df[df:MatrixInverse]
\ExecuteMetaData[../map4.tex]{df:MatrixInverse}
\end{frame}




%..........
\begin{frame}
\lm[le:LeftAndRightInvEqual]
\ExecuteMetaData[../map4.tex]{lm:LeftAndRightInvEqual}
\pause
\th[th:MatrixInvertibleIffNonsingular]
\ExecuteMetaData[../map4.tex]{th:MatrixInvertibleIffNonsingular}
\pause
\pf
\ExecuteMetaData[../map4.tex]{pf:MatrixInvertibleIffNonsingular}
\qed
\end{frame}




%..........
\begin{frame}
\lm[lem:ProdInvIsInv]
\ExecuteMetaData[../map4.tex]{lm:ProdInvIsInv}
\pause
\pf
\ExecuteMetaData[../map4.tex]{pf:ProdInvIsInv0}

\pause
\ExecuteMetaData[../map4.tex]{pf:ProdInvIsInv1}
\qed

\pause
Here is the arrow diagram for matrix inverses.
\centergraphic{../ch3.21}
\end{frame}




\begin{frame}
\lm[lem:ComputeInvMat]
\ExecuteMetaData[../map4.tex]{lm:ComputeInvMat}
\pause
\pf
\ExecuteMetaData[../map4.tex]{pf:ComputeInvMat0}

\pause
\ExecuteMetaData[../map4.tex]{pf:ComputeInvMat1}

\pause
\ExecuteMetaData[../map4.tex]{pf:ComputeInvMat2}
\qed  
\end{frame}




\begin{frame}
\co[cor:TwoByTwoInv]
\ExecuteMetaData[../map4.tex]{co:TwoByTwoInv}
\pause
\pf
\ExecuteMetaData[../map4.tex]{pf:TwoByTwoInv}
\qed  
\end{frame}


%...........................
% \begin{frame}
% \ExecuteMetaData[../gr3.tex]{GaussJordanReduction}
% \df[def:RedEchForm]
% 
% \end{frame}
\end{document}
