%\documentstyle [11pt,amsfonts]{mybook}
%\input{latexmac}

%\setcounter{chapter}{0}
%\setcounter{section}{0}
%\setcounter{subsection}{0}
%
%\begin{document}
%\pagestyle{empty}
\thispagestyle{empty}
% \vfill
%\medskip
\begin{center}
  \textbf{Summary of notation}  \\[2ex]
  \begin{tabular}{r|l}
    \( \Re \), \( \Re^+ \), \( \Re^n \) &real numbers, reals greater than $0$, $n$-tuples of reals \\
    \( \N              \),
    \( \C              \)  &natural numbers: \( \set{0,1,2,\ldots} \), complex numbers                           \\
    \( (a\,..\,b) \), \( [a\,..\,b] \) &interval (open, closed) of reals between $a$ and $b$  \\
    \( \sequence{\ldots} \)&sequence; like a set but order matters    \\
    \( V,W,U \)            &vector spaces                             \\
    \( \vec{v},\vec{w} \),
    $\zero$, $\zero_V$     &vectors, zero vector, zero vector of $V$   \\
    \( B,D \), \( \vec{\beta},\vec{\delta} \)         
                          &bases, basis vectors                      \\
    \( \stdbasis_n=\sequence{\vec{e}_1,\,\ldots,\,\vec{e}_n} \)      
                          &standard basis for $\Re^n$                \\
    \( \rep{\vec{v}}{B} \) &matrix representing the vector            \\
    \( \polyspace_n \)     &set of degree \( n \) polynomials      \\
    \( \matspace_{\nbym{n}{m}} \)  &set of \( \nbym{n}{m} \) matrices    \\
    \( \spanof{S} \)       &span of the set \( S \)                   \\
    \( M\directsum N \)    &direct sum of subspaces                   \\
    \( V\isomorphicto W \) &isomorphic spaces                         \\
    \( h,g \)              &homomorphisms, linear maps                \\
    \( H,G \)              &matrices                                  \\
    \( t,s \)              &transformations; maps from a space to itself \\
    \( T,S \)              &square matrices                           \\
    \( \rep{h}{B,D} \)     &matrix representing the map \( h \)       \\
    \( h_{i,j} \)          &matrix entry from row \( i \),
                              column \( j \)                      \\
    \( Z_{\nbym{n}{m}},Z,I_{\nbyn{n}},I \)        &zero matrix, identity matrix    \\
    \( \deter{T} \)        &determinant of the matrix \( T \)         \\
    \( \rangespace{h},\nullspace{h} \)
                           &rangespace and nullspace of the map \( h \) \\
    \( \genrangespace{h},\gennullspace{h} \)
                           &generalized rangespace and nullspace
  \end{tabular}
\end{center}\vfill
\begin{center}
  \textbf{Lower case Greek alphabet, with pronounciation}
    \\[1ex]
  \newcommand{\pronounced}[1]{\hspace*{.2em}\small\textit{#1}}
  \begin{tabular}{lc@{\hspace*{3em}}lc}
    \multicolumn{1}{c}{name}    &shape      
    &\multicolumn{1}{c}{name}   &shape    \\ 
    \hline
     Alpha \pronounced{AL-fuh}  &\( \alpha  \)  
       &Nu  \pronounced{NEW}    &\( \nu     \)   \\
     Beta  \pronounced{BAY-tuh}   &\( \beta   \)  
       &Xi   \pronounced{KSIGH}   &\( \xi  \) \\ 
     Gamma  \pronounced{GAM-muh}  &\( \gamma  \)  
       &Omicron  \pronounced{OM-uh-CRON} &\( o       \) \\
     Delta  \pronounced{DEL-tuh}  &\( \delta  \)  
       &Pi  \pronounced{PIE}    &\( \pi \) \\
     Epsilon  \pronounced{EP-suh-lon} &\( \epsilon\)  
       &Rho  \pronounced{ROW}   &\( \rho \) \\
     Zeta   \pronounced{ZAY-tuh}  &\( \zeta   \)  
       &Sigma  \pronounced{SIG-muh} &\( \sigma  \) \\
     Eta  \pronounced{AY-tuh}    &\( \eta    \)  
       &Tau  \pronounced{TOW to rhyme with cow}   &\( \tau \) \\
     Theta  \pronounced{THAY-tuh}  &\( \theta  \)  
       &Upsilon  \pronounced{OOP-suh-LON} &\( \upsilon\) \\
     Iota \pronounced{eye-OH-tuh}  &\( \iota \) 
       &Phi  \pronounced{FEE, or FI to rhyme with hi}   &\( \phi    \) \\
     Kappa  \pronounced{KAP-uh} &\( \kappa  \) 
       &Chi  \pronounced{KI to rhyme with hi}   &\( \chi    \) \\
     Lambda  \pronounced{LAM-duh} &\( \lambda \) 
       &Psi \pronounced{SIGH, or PSIGH}   &\( \psi    \) \\
     Mu  \pronounced{MEW}    &\( \mu  \) 
       &Omega  \pronounced{oh-MAY-guh}  &\( \omega  \)
  \end{tabular}
\end{center}
\vfill
\par\noindent{\small\textbf{Cover.}
  This is Cramer's Rule for the system $x_1+2x_2=6$, 
  $3x_1+x_2=8$.
  The size of the first box is the given determinant.
  The size of the second box is $x_1$ times that, and equals the size
  of the final box.
  Hence, $x_1$ is the final determinant divided by the first.}
%\end{document}
