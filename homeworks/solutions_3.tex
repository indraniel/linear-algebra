\documentclass[11pt]{article}
\usepackage[margin=1in]{geometry}
\usepackage{../linalgjh}

\setlength{\parindent}{0em}
\pagestyle{empty}
\begin{document}\thispagestyle{empty}
\makebox[\linewidth]{\textbf{Homework, MA~213}\hspace*{4in}\textbf{Due: 2014-Sep-29}}

\begin{enumerate}
\item
Find a basis for each space.
Verify that it is a basis.
  \begin{enumerate}
  \item The subspace $M=\set{a+bx+cx^2+dx^3\suchthat a-2b+c-d=0}$ 
   of $\polyspace_3$.

   Parametrize $a-2b+c-d=0$ as $a=2b-c+d$, $b=b$, $c=c$, and~$d=d$ to get
   this description of $M$ as the span of a set of three vectors.
   \begin{equation*}
     M=\set{
      (2+x)\cdot b+(-1+x^2)\cdot c+(1+x^3)\cdot d
      \suchthat b,c,d\in\Re
      }
   \end{equation*}
   To show that this three-vector set is a basis, what remains is for us to
   verify that it is linearly independent.
   \begin{equation*}
      0+0x+0x^2+0x^3=(2+x)\cdot c_1+(-1+x^2)\cdot c_2+(1+x^3)\cdot c_3
   \end{equation*}
   From the $x$~terms we see that $c_1=0$.
   From the $x^2$~terms we see that $c_2=0$.
   The $x^3$~terms give that $c_3=0$. 

  \item This subspace of $\matspace_{\nbyn{2}}$.
    \begin{equation*}
      W=\set{
        \begin{mat}
          a  &b  \\
          c  &d
        \end{mat}
        \suchthat a-c=0}
    \end{equation*}

  First parametrize the description
  (note that the fact that $b$ and~$d$ are not mentioned in the description
  of $W$ does not mean
  they are zero or absent, it means that they are unrestricted).
  \begin{equation*}
    W=\set{
      \begin{mat}
        0 &1 \\
        0 &0
      \end{mat}\cdot b
      +
      \begin{mat}
        1 &0 \\
        1 &0
      \end{mat}\cdot c
      +
      \begin{mat}
        0 &0 \\
        0 &1
      \end{mat}\cdot d
      \suchthat b,c,d\in\Re}
  \end{equation*}
  That gives $W$ as the span of a three element set.
  We will be done if we show that the set is linearly independent.
  \begin{equation*}
      \begin{mat}
        0 &0 \\
        0 &0
      \end{mat}
      =
      \begin{mat}
        0 &1 \\
        0 &0
      \end{mat}\cdot c_1
      +
      \begin{mat}
        1 &0 \\
        1 &0
      \end{mat}\cdot c_2
      +
      \begin{mat}
        0 &0 \\
        0 &1
      \end{mat}\cdot c_3   
  \end{equation*}
  Using the upper right entries we see that $c_1=0$.
  The upper left entries give that $c_2=0$, and the lower left entries 
  show that $c_3=0$.
  \end{enumerate}

\item Give two different bases for $\Re^3$.
  Verify that each is a basis.

  Obviously there are many different correct choices of bases.
  The natural basis for $\Re^3$ is this.
  \begin{equation*}
    \stdbasis_3=\sequence{\colvec{1 \\ 0 \\ 0},
                          \colvec{0 \\ 1 \\ 0},
                          \colvec{0 \\ 0 \\ 1}}
  \end{equation*}
  The verification that it spans $\Re^3$ is easy: for any $x,y,z\in\Re$
  this equation has a solution,
  \begin{equation*}
    \colvec{x \\ y \\ z}
      =\colvec{1 \\ 0 \\ 0 }\cdot c_1
       +\colvec{0 \\ 1 \\ 0}\cdot c_2
       +\colvec{0 \\ 0 \\ 1}\cdot c_3
     \tag{$*$}
  \end{equation*}
  namely, $c_1=x$, $c_2=y$, and~$c_3=z$.
  Further, the set is linearly independent since the relationship
  \begin{equation*}
    \colvec{0 \\ 0 \\ 0}
      =\colvec{1 \\ 0 \\ 0 }\cdot c_1
       +\colvec{0 \\ 1 \\ 0}\cdot c_2
       +\colvec{0 \\ 0 \\ 1}\cdot c_3    
  \end{equation*}
  obviously has only the trivial solution.
  (\textit{Comment.}
   We could have done the argument in one step by observing that 
   equation~($*$) shows that 
   each vector from $\Re^3$ is represented with respect to
   this basis~$\stdbasis_3$ in one and only one way.)

  This is a second basis for $\Re^3$.
  \begin{equation*}
    B=\sequence{\colvec{1 \\ 0 \\ 0},
                \colvec{1 \\ 1 \\ 0},
                \colvec{1 \\ 1 \\ 1}}
  \end{equation*}
  To verify that that it spans $\Re^3$ suppose $x,y,z\in\Re$,
  then
  \begin{equation*}
    \colvec{x \\ y \\ z}
      =\colvec{1 \\ 0 \\ 0 }\cdot c_1
       +\colvec{1 \\ 1 \\ 0}\cdot c_2
       +\colvec{1 \\ 1 \\ 1}\cdot c_3
  \end{equation*}
  has a solution,
  namely $c_1=x-y$, $c_2=y-z$, and~$c_3=z$.
  Further, the set~$B$ is linearly independent since in the relationship
  \begin{equation*}
    \colvec{0 \\ 0 \\ 0}
      =\colvec{1 \\ 0 \\ 0 }\cdot c_1
       +\colvec{1 \\ 1 \\ 0}\cdot c_2
       +\colvec{1 \\ 1 \\ 1}\cdot c_3    
  \end{equation*}
  the third components give that $c_3=0$, then the second components 
  give that $c_2=0$, and with those two the first components give that
  $c_1=0$.

\item Represent the vector with respect to each of the two bases.
  \begin{equation*}
    \vec{v}=\colvec{3  \\ -1}
    \quad
    B_1=\sequence{\colvec{1  \\ -1}, \colvec{1  \\ 1}},\;
    B_2=\sequence{\colvec{1 \\ 2}, \colvec{1 \\ 3}}
  \end{equation*}

  Solving
  \begin{equation*}
    \colvec{3 \\ -1}=\colvec{1 \\ -1}\cdot c_1
                     +\colvec{1 \\ 1}\cdot c_2
  \end{equation*}
  gives $c_1=2$ and~$c_2=1$. 
  \begin{equation*}
    \rep{\colvec{3 \\ -1}}{B_1}
       =\colvec{2 \\ 1}_{B_1}
  \end{equation*}
  Similarly, solving
  \begin{equation*}
    \colvec{3 \\ -1}=\colvec{1 \\ 2}\cdot c_1
                     +\colvec{1 \\ 3}\cdot c_2
  \end{equation*}
  gives this. 
  \begin{equation*}
    \rep{\colvec{3 \\ -1}}{B_2}
       =\colvec{10 \\ -7}_{B_2}
  \end{equation*}
\end{enumerate}
\end{document}
