\chapter{Geometry of Linear Maps}

\Sage{} can illustrate the geometric effect of linear transformations.
In this chapter we will picture their effect on the unit square.
The graphs here show transformations of the plane~$\Re^2$
because those fit on the paper but the principles extend to 
higher-dimensional spaces.




%========================================
\section{Lines map to lines}
The prior chapter points out that 
the condition $h(r\vec{v})=r\cdot h(\vec{v})$ in the definition
of linear map means that
linear maps send lines through the origin to lines through the origin.
We can extend that to show that under a linear map, the image of a line
in the domain is a line in the range.
Fix a domain space~$\Re^d$ and codomain space~$\Re^c$, along with
the linear map~$h$ between them.
Get a line in the domain by fixing a vector of slopes~$\vec{m}\in\Re^d$
and a vector of offsets from the origin~$\vec{b}\in\Re^d$, so  
$\ell = \set{\vec{v}=\vec{m}\cdot s +\vec{b}\suchthat s\in\Re}$.
The image of~$\ell$ is this set.
\begin{equation*}
  h(\ell)=\set{h(\vec{m}\cdot s+\vec{b})\suchthat s\in\Re}
  =\set{h(\vec{m})\cdot s+h(\vec{b})\suchthat s\in\Re}
\end{equation*}
This is a line in the codomain with the vector of 
slopes~$h(\vec{m})$ and the vector of offsets~$h(\vec{b})$. 

For example, consider the transformation $\map{t}{\Re^2}{\Re^2}$ 
that rotates vectors counterclockwise by $\pi/6$~radians.
\begin{equation*}
  \rep{t}{\stdbasis_2,\stdbasis_2}
  =
  \begin{mat}
    \cos(\pi/6)  &\sin(\pi/6) \\
    -\sin(\pi/6)  &\cos(\pi/6)
  \end{mat}
  = 
  \begin{mat}
    \sqrt{3}/2   &1/2 \\
    -1/2          &\sqrt{3}/2
  \end{mat}
\end{equation*}
This is the line $y=3x+2$.
\begin{equation*}
  \ell=\set{\colvec{x \\ y}=\colvec{3 \\ 1}\cdot s+\colvec{2 \\ 0}\suchthat s\in\Re}
\end{equation*}
Under the action of~$t$ that line becomes this set.
\begin{equation*}
  t(\ell)
  =
  \set{\colvec{x \\ y}=\frac{1}{2}\colvec{3\sqrt{3}-1 \\ 3+\sqrt{3}}\cdot s
                                  +\colvec{\sqrt{3} \\ 1}\suchthat s\in\Re}
\end{equation*}
\begin{sageoutput}[d,1,6]
s = var('s')
plot.options['figsize'] = 2.5
plot.options['axes_pad'] = 0.05
plot.options['fontsize'] = 7
plot.options['dpi'] = 1200
plot.options['aspect_ratio'] = 1
ell = parametric_plot((3*s+2,1*s), (s, -10, 10))
ell.set_axes_range(-4, 4, -4, 4)
ell.save("graphics/geo000a.png")
t_x(s) = ((3*sqrt(3)-1)/2)*s+sqrt(3)
t_y(s) = ((3+sqrt(3))/2)*s+1
t_ell = parametric_plot((t_x(s), t_y(s)), (s, -10, 10))
t_ell.set_axes_range(-4, 4, -4, 4)
t_ell.save("graphics/geo000b.png")
\end{sageoutput}
\begin{sagesilent}
s = var('s')
plot.options['figsize'] = 2.5
plot.options['axes_pad'] = 0.05
plot.options['fontsize'] = 7
plot.options['dpi'] = 1200
plot.options['aspect_ratio'] = 1
ell = parametric_plot((3*s+2,1*s), (s, -10, 10))
ell.set_axes_range(-4, 4, -4, 4)
ell.save("graphics/geo000a.png", fontsize=7)
t_x(s) = ((3*sqrt(3)-1)/2)*s+sqrt(3)
t_y(s) = ((3+sqrt(3))/2)*s+1
t_ell = parametric_plot((t_x(s), t_y(s)), (s, -10, 10))
t_ell.set_axes_range(-4, 4, -4, 4)
t_ell.save("graphics/geo000b.png", fontsize=7)
\end{sagesilent}
\begin{equation*}
  \vcenteredhbox{\includegraphics{graphics/geo000a.png}}
  \quad\mapsvia{t}\quad
  \vcenteredhbox{\includegraphics{graphics/geo000b.png}}
\end{equation*}
(The limits on the parameter~$s$ of $10$ and~$-10$ are arbitrary, just
chosen to be large enough that the line segment covers the entire 
domain and codomain intervals shown, from $-4$ to~$4$.)
One instance is that the vector that ends at $(2,0)$ is rotated to the vector
that ends at $(\sqrt{3},1)$.




%========================================
\section{The unit square}
We can see the effect of plane transformations 
$\map{t}{\Re^2}{\Re^2}$ 
by applying them to
this unit square.
\begin{sagesilent}
load "plot_action.sage"
p = plot_square_action(1,0,0,1)  # identity matrix
p.set_axes_range(-0.25, 1.5, -0.25, 1.5) 
p.save("graphics/geo100.png", dpi=1200)
\end{sagesilent}
\begin{center}
  \includegraphics{graphics/geo100.png}
\end{center}
That graph was generated by this \Sage{} session.
\begin{sageoutput}
load "plot_action.sage"
p = plot_square_action(1,0,0,1)  # identity matrix
p.set_axes_range(-0.25, 1.5, -0.25, 1.5) 
p.save("graphics/geo100.png")
\end{sageoutput}
\noindent The \inlinecode{plot_square_action(a,b,c,d)} 
routine plots the action on the square of the 
transformation represented with respect to the standard basis by the 
matrix with entries $a$, $b$, $c$, and~$d$.
\begin{equation*}
  % \rep{t}{\stdbasis_2,\stdbasis_2}=
  \rowvec{x  &y}
  \begin{mat}
    a &b \\
    c &d
  \end{mat}
  =
  \rowvec{ax+cy &bx+dy}
\end{equation*}
The routine's source is at the end of this chapter but
the observation above  
that linear maps send lines to lines makes it easy:
the matrix 
has this action on the four corners of the square 
\begin{equation*}
  \colvec{0 \\ 0}\mapsunder{t}\colvec{0 \\ 0}\quad
  \colvec{1 \\ 0}\mapsunder{t}\colvec{a \\ b}\quad
  \colvec{1 \\ 1}\mapsunder{t}\colvec{a+c \\ b+d}\quad
  \colvec{0 \\ 1}\mapsunder{t}\colvec{c \\ d}
\end{equation*}
and the routine plots the four line segments connecting those corners.
In the \Sage{} session above, that routine is given the identity matrix,
so it plots the unit square unchanged.

That routine gives you
these pictures of the effect of the generic matrix.
\begin{sageoutput}
load "plot_action.sage"
q = plot_square_action(1,0,0,1) 
q.set_axes_range(-6, 6, -1, 6) 
q.save("graphics/geo101a.png")
p = plot_square_action(1,2,3,4) 
p.set_axes_range(-6, 6, -1, 6) 
p.save("graphics/geo101b.png")
\end{sageoutput}
\begin{sagesilent}
load "plot_action.sage"
q = plot_square_action(1,0,0,1) 
q.set_axes_range(-6, 6, -1, 6) 
q.save("graphics/geo101a.png", dpi=1200)
p = plot_square_action(1,2,3,4) 
p.set_axes_range(-6, 6, -1, 6) 
p.save("graphics/geo101b.png", dpi=1200)
\end{sagesilent}
\begin{equation*}
  \vcenteredhbox{\includegraphics{graphics/geo101a.png}}
  \quad\mapsvia{\big (\begin{smallmatrix} 1 &2 \\ 3 &4 \end{smallmatrix}\big )}\quad
  \vcenteredhbox{\includegraphics{graphics/geo101b.png}}
\end{equation*}
The colors show another effect of transformations, beyond shape-changing
while line-preserving.
Take the colors in their natural order of red, orange, 
green, and blue.
Then the domain square has a counterclockwise orientation while the codomain's
figure is clockwise, so the colors illustrate whether the transformation 
preserves or reverses orientation.

We can develop an understanding of complex behavior by 
building on an understanding of simple behavior.
This transformation doubles the $x$~components of all 
vectors.\footnote{In the rest of this section we'll omit the fiddly
lines that load the script file, set the axis ranges, etc.} 
\begin{sageoutput}[d,0,4;d,5,7]
load "plot_action.sage"
q = plot_square_action(1,0,0,1) 
q.set_axes_range(-4, 4, -1, 4) 
q.save("graphics/geo102a.png")
p = plot_square_action(2,0,0,1)  
p.set_axes_range(-4, 4, -1, 4) 
p.save("graphics/geo102b.png")
\end{sageoutput}
\begin{sagesilent}
load "plot_action.sage"
q = plot_square_action(1,0,0,1) 
q.set_axes_range(-4, 4, -1, 4) 
q.save("graphics/geo102a.png",dpi=1200)
p = plot_square_action(2,0,0,1)  
p.set_axes_range(-4, 4, -1, 4) 
p.save("graphics/geo102b.png",dpi=1200)
\end{sagesilent}
\begin{equation*}
  \vcenteredhbox{\includegraphics{graphics/geo102a.png}}
  \quad\mapsvia{\big (\begin{smallmatrix} 2 &0 \\ 0 &1 \end{smallmatrix}\big )}\quad
  \vcenteredhbox{\includegraphics{graphics/geo102b.png}}
  \tag{$*$}
\end{equation*}
\noindent
Linear maps send the zero vector to the zero vector, and the input
square is anchored at the origin, so 
the output shape is also anchored at the origin.
But it has been stretched horizontally\Dash it has the same orientation
as the starting square, but twice the area.

That example illustrates that the behavior associated with diagonal matrices
is simple.
For instance, tripling the $x$~coordinate gives you a similar shape with
three times the area of the starting one.
What about if you take $-1$~times the $x$-coordinate?
\begin{sageoutput}[d,0,4;d,5,7]
load "plot_action.sage"
q = plot_square_action(1,0,0,1) 
q.set_axes_range(-4, 4, -1, 4) 
q.save("graphics/geo103a.png")
p = plot_square_action(-1,0,0,1) 
p.set_axes_range(-4, 4, -1, 4) 
p.save("graphics/geo103b.png")
\end{sageoutput}
\begin{sagesilent}
load "plot_action.sage"
q = plot_square_action(1,0,0,1) 
q.set_axes_range(-4, 4, -1, 4) 
q.save("graphics/geo103a.png",dpi=1200)
p = plot_square_action(-1,0,0,1) 
p.set_axes_range(-4, 4, -1, 4) 
p.save("graphics/geo103b.png",dpi=1200)
\end{sagesilent}
\begin{equation*}
  \vcenteredhbox{\includegraphics{graphics/geo103a.png}}
  \quad\mapsvia{\big (\begin{smallmatrix} -1 &0 \\ 0 &1 \end{smallmatrix}\big )}\quad
  \vcenteredhbox{\includegraphics{graphics/geo103b.png}}
  \tag{$**$}
\end{equation*}
\noindent
It changes the orientation.

We say that the above shape has an \textit{oriented area}
of $-1$.
The motivation for taking the area with a sign 
is:~imagine starting with the right-hand figure 
from the example before this one
and slide the orange side in from the right, from $2$ to~$1$, to~$0$ and
then to~$-1$.
The area falls from~$2$ to~$1$, to~$0$, and so we naturally
to assign the figure above an area measure of~$-1$.
The prefix `oriented' is just there to distinguish this idea from the
grade school meaning of
area.
That meaning of area is the absolute value of the oriented area.

The next transformation combines action in two axes, 
tripling the $y$~components and multiplying 
$x$~components by $-1$. 
\begin{sageoutput}[d,0,4;d,5,7]
load "plot_action.sage"
q = plot_square_action(1,0,0,1) 
q.set_axes_range(-4, 4, -1, 4) 
q.save("graphics/geo104a.png")
p = plot_square_action(-1,0,0,3) 
p.set_axes_range(-4, 4, -1, 4) 
p.save("graphics/geo104b.png")
\end{sageoutput}
\begin{sagesilent}
load "plot_action.sage"
q = plot_square_action(1,0,0,1) 
q.set_axes_range(-4, 4, -1, 4) 
q.save("graphics/geo104a.png",dpi=1200)
p = plot_square_action(-1,0,0,3) 
p.set_axes_range(-4, 4, -1, 4) 
p.save("graphics/geo104b.png",dpi=1200)
\end{sagesilent}
\begin{equation*}
  \vcenteredhbox{\includegraphics{graphics/geo104a.png}}
  \quad\mapsvia{\big (\begin{smallmatrix} -1 &0 \\ 0 &3 \end{smallmatrix}\big )}\quad
  \vcenteredhbox{\includegraphics{graphics/geo104b.png}}
\end{equation*}
The colors show that this transformation also changes
the orientation, so the new shape has an oriented area of~$-3$.

What if we change the orientation twice?
\begin{sageoutput}[d,0,4;d,5,7]
load "plot_action.sage"
q = plot_square_action(1,0,0,1) 
q.set_axes_range(-4, 4, -4, 2) 
q.save("graphics/geo105a.png")
p = plot_square_action(-2,0,0,-3) 
p.set_axes_range(-4, 4, -4, 2) 
p.save("graphics/geo105b.png")
\end{sageoutput}
\begin{sagesilent}
load "plot_action.sage"
q = plot_square_action(1,0,0,1) 
q.set_axes_range(-4, 4, -4, 2) 
q.save("graphics/geo105a.png",dpi=1200)
p = plot_square_action(-2,0,0,-3) 
p.set_axes_range(-4, 4, -4, 2) 
p.save("graphics/geo105b.png",dpi=1200)
\end{sagesilent}
\begin{equation*}
  \vcenteredhbox{\includegraphics{graphics/geo105a.png}}
  \quad\mapsvia{\big (\begin{smallmatrix} -2 &0 \\ 0 &-3 \end{smallmatrix}\big )}\quad
  \vcenteredhbox{\includegraphics{graphics/geo105b.png}}
\end{equation*}
The colors are the same as the original shape's counterclockwise
red, orange, green, and then blue.
Thus the new shape has an oriented area of~$6$.

We next show the effect of putting in off-diagonal entries.
\begin{sageoutput}[d,0,4;d,5,7]
load "plot_action.sage"
q = plot_square_action(1,0,0,1) 
q.set_axes_range(-4, 4, -1, 4) 
q.save("graphics/geo106a.png")
p = plot_square_action(1,0,2,1) 
p.set_axes_range(-4, 4, -1, 4) 
p.save("graphics/geo106b.png")
\end{sageoutput}
\begin{sagesilent}
load "plot_action.sage"
q = plot_square_action(1,0,0,1) 
q.set_axes_range(-4, 4, -1, 4) 
q.save("graphics/geo106a.png",dpi=1200)
p = plot_square_action(1,0,2,1) 
p.set_axes_range(-4, 4, -1, 4) 
p.save("graphics/geo106b.png",dpi=1200)
\end{sagesilent}
\begin{equation*}
  \vcenteredhbox{\includegraphics{graphics/geo106a.png}}
  \quad\mapsvia{\big (\begin{smallmatrix} 1 &0 \\ 2 &1 \end{smallmatrix}\big )}\quad
  \vcenteredhbox{\includegraphics{graphics/geo106b.png}}
\end{equation*}
This transformation is a \textit{skew}\footnote{The word skew means ``to distort from a symmetrical form.''}
(or \textit{shear}).
The line segment sides of the original square 
map to line segments, but the sides are not at right angles.
The action
\begin{equation*}
  \colvec{x \\ y} \mapsto \colvec{x+2y \\ y}
\end{equation*}
means that 
a starting vector with a $y$~component of~$1$ gets shifted right by~$2$ while
a starting vector with a $y$~component of~$2$ is shifted right by~$4$, so
vectors are shifted depending on how far they are above or below the
$x$-axis.
This transformation preserves orientation and the output shape has a base of~$1$
with a height of~$1$ so its oriented area is~$1$.

Putting a nonzero value in the other off-diagonal entry of the matrix,
the upper right, has the same effect except that it skews parallel
to the $y$-axis.
\begin{sageoutput}[d,0,4;d,5,7]
load "plot_action.sage"
q = plot_square_action(1,0,0,1) 
q.set_axes_range(-4, 4, -3, 2) 
q.save("graphics/geo107a.png")
p = plot_square_action(1,-2,0,1) 
p.set_axes_range(-4, 4, -3, 2) 
p.save("graphics/geo107b.png")
\end{sageoutput}
\begin{sagesilent}
load "plot_action.sage"
q = plot_square_action(1,0,0,1) 
q.set_axes_range(-4, 4, -3, 2) 
q.save("graphics/geo107a.png")
p = plot_square_action(1,-2,0,1) 
p.set_axes_range(-4, 4, -3, 2) 
p.save("graphics/geo107b.png")
\end{sagesilent}
\begin{equation*}
  \vcenteredhbox{\includegraphics{graphics/geo107a.png}}
  \quad\mapsvia{\big (\begin{smallmatrix} 1 &-2 \\ 0 &1 \end{smallmatrix}\big )}\quad
  \vcenteredhbox{\includegraphics{graphics/geo107b.png}}
\end{equation*}
The action
\begin{equation*}
  \colvec{x \\ y} \mapsto \colvec{x \\ -2x+y}
\end{equation*}
means that vectors are shifted depending on how far they are from the
$x$~axis.
For instance, a vector with an $x$~component of~$1$ is shifted by~$-2$.
The oriented area of the output shape is~$1$.





\section{Determinants}
The book geometrically 
interprets the conditions in the definition of 
a determinant function.
It shows that, in going from the before picture to the after, 
these transformation matrices 
change the oriented area of the input region by a factor that is the
determinant of the matrix.   
In ($*$) the matrix has determinant~$2$ and it doubles the oriented area.
In ($**$) the matrix multiplies the oriented area by~$-1$.

One advantage of starting these before/after pictures with a unit
square is that then the output shape has oriented area equal to the
determinant of the matrix. 

For instance, this matrix is singular, so it has determinant~$0$.
\begin{sageoutput}[d,0,4;d,5,7]
load "plot_action.sage"
q = plot_square_action(1,0,0,1) 
q.set_axes_range(-4, 4, -1, 6) 
q.save("graphics/geo200a.png")
p = plot_square_action(1,2,2,4) 
p.set_axes_range(-4, 4, -1, 4) 
p.save("graphics/geo200b.png")
\end{sageoutput}
\begin{sagesilent}
load "plot_action.sage"
q = plot_square_action(1,0,0,1) 
q.set_axes_range(-4, 4, -1, 6) 
q.save("graphics/geo200a.png",dpi=1200)
p = plot_square_action(1,2,2,4) 
p.set_axes_range(-4, 4, -1, 6) 
p.save("graphics/geo200b.png",dpi=1200)
\end{sagesilent}
\begin{equation*}
  \vcenteredhbox{\includegraphics{graphics/geo200a.png}}
  \quad\mapsvia{\big (\begin{smallmatrix} 1 &2 \\ 2 &4 \end{smallmatrix}\big )}\quad
  \vcenteredhbox{\includegraphics{graphics/geo200b.png}}
\end{equation*}
\noindent
The output shape has zero content.






\section{Turing's factorization PA=LDU}
We will now see how the action of any matrix can be decomposed into 
the actions shown above.
This will give us a complete geometric description of any linear map,
that is, you can understand the effect of any transformation by 
breaking down into a sequence of steps that are simple.

Recall that you can do the row operations of Gauss's Method with
matrix multiplication.
For instance, multiplication from the left by this matrix has the effect of the
row operation $2\rho_1+\rho_2$.
\begin{equation*}
  \begin{mat}
    1 &0 &0 \\
    2 &1 &0 \\
    0 &0 &1
  \end{mat}
  \begin{mat}
    3 &1 &4 \\
   -6 &1 &-8 \\
    0 &-3 &2
  \end{mat}
  =
  \begin{mat}
    3 &1  &4 \\ 
    0 &3 &0 \\
    0 &-3  &2
  \end{mat}
\end{equation*}

In general, as described in the book, the  
\definend{elementary reduction matrices}
come in three types $M_i(k)$, $P_{i,j}$, and~$C_{i,j}(k)$, and
arise from applying a row operation to an identity matrix.
\begin{center}
$I\grstep{k\rho_i}M_i(k)$ for \( k\neq 0 \)
\qquad
\( I\grstep{\rho_i\leftrightarrow\rho_j}P_{i,j} \) for \( i\neq j \)
\qquad
\( I\grstep{k\rho_i+\rho_j}C_{i,j}(k) \) for \( i\neq j \)
\end{center}
For a matrix~$H$ you can do row scaling~\( k\rho_i \) 
with \( M_i(k)H \), 
you can swap rows \( \rho_i\leftrightarrow\rho_j \) with \( P_{i,j}H \), 
and you can add a multiple of one row to another
\( k\rho_i+\rho_j \) with \( C_{i,j}(k)H \). 
The prior paragraph used the $\nbyn{3}$
matrix $C_{1,2}(2)$.
(We are focused on transformations so we will assume all of these
matrices are same-sized and square.)

You can continue the Gauss's Method started in the equation above
and use $C_{2,3}(-1)$ to perform $-\rho_2+\rho_3$,
producing echelon form.
\begin{equation*}
  \begin{mat}
    1 &0  &0 \\
    0 &1  &0 \\
    0 &-1 &1
  \end{mat}
  \begin{mat}
    1 &0 &0 \\
    2 &1 &0 \\
    0 &0 &1
  \end{mat}
  \begin{mat}
    3 &1 &4 \\
   -6 &1 &-8 \\
    0 &-3 &2
  \end{mat}
  =
  \begin{mat}
    3 &1  &4 \\ 
    0 &3  &0 \\
    0 &0  &2
  \end{mat}
  \tag{$*$}
\end{equation*}
As in this example, matrix multiplication by these elementary matrices 
suffices to do Gauss's Method and produce echelon form.

Observe further that if the starting matrix is such that
you don't need any row swapping then 
you can stick with the operations~\( k\rho_i+\rho_j \) where 
$j>i$ (that is, where a row operates on a row below it). 
The elementary
matrices that perform those operations are \textit{lower triangular}
since all of their nonzero entries are in the lower left.
(Matrices with all of their nonzero entries in the upper right are 
\textit{upper triangular}.)
Thus your factorization so far is into a product of lower triangular elementary
matrices and an echelon form matrix.

You can go further.
Use a diagonal matrix to make the 
leading entries of the nonzero rows of the echelon form matrix into~$1$'s.
Here is that additional step performed on the above equation.
\begin{equation*}
  \begin{mat}
    1/3 &0   &0 \\
    0   &1/3 &0 \\
    0   &0   &1/2  
  \end{mat}
  \begin{mat}
    1 &0  &0 \\
    0 &1  &0 \\
    0 &-1 &1
  \end{mat}
  \begin{mat}
    1 &0 &0 \\
    2 &1 &0 \\
    0 &0 &1
  \end{mat}
  \begin{mat}
    3 &1 &4 \\
   -6 &1 &-8 \\
    0 &-3 &2
  \end{mat}
  =
  \begin{mat}
    1 &1/3  &4/3 \\ 
    0 &1  &0 \\
    0 &0  &1
  \end{mat}
  \tag{$*$}
\end{equation*}
 
You can use multiplication by elementary matrices to go all the way to a 
block partial identity
matrix.
The idea is to use column operations.
Here is right-multiplication on the right-hand side of~($*$) to 
add $-1/3$ times the first column to the second column.
\begin{equation*}
  \begin{mat}
    1 &1/3  &4/3 \\ 
    0 &1  &0 \\
    0 &0  &1
  \end{mat}
  \begin{mat}
    1  &-1/3  &0  \\
    0  &1     &0  \\
    0  &0     &1
  \end{mat}
  =
  \begin{mat}
    1 &0  &4 \\ 
    0 &1  &0 \\
    0 &0  &1
  \end{mat}
\end{equation*}
Then adding $-4/3$~times the first column to the third column
leave an identity matrix.
\begin{equation*}
  \begin{mat}
    3 &1  &4 \\ 
    0 &3  &0 \\
    0 &0  &2
  \end{mat}
  \begin{mat}
    1  &0     &-4/3  \\
    0  &1     &0  \\
    0  &0     &1
  \end{mat}
  =
  \begin{mat}
    1 &0  &0 \\ 
    0 &1  &0 \\
    0 &0  &1
  \end{mat}
   \tag{$**$}
\end{equation*}
Thus, if you start with a matrix $A$ that does not require any row swaps then 
you get this matrix equation.
\begin{equation*}
  L_1L_2\cdots L_k\cdot A\cdot U_1U_2\cdot U_r = D
\end{equation*}
Here $D$ is a partial identity matrix, 
the $L_i$ are lower-triangular row combination matrices,
and the $U_j$ are upper-triangular column combination matrices. 

All of the row operations can be undone
(for instance, $2\rho_1+\rho_2$
is undone with $-2\rho_1+\rho_2$), 
Thus each of those lower triangular matrices
has an inverse.
Likewise, each upper-triangular matrix has an inverse.
Therefore, 
if you don't need any swaps in a Gauss-Jordan reduction of a matrix~$A$ 
then you have this factorization.
\begin{equation*} 
  A = L_k^{-1}\cdots L_1^{-1}\cdot D\cdot U_r^{-1}\cdots U_1^{-1}
\end{equation*}
To fix the swap issue you can pre-swap:
before factoring the starting matrix, first swap its rows
with a permutation matrix~$P$.
\begin{equation*}
  P\cdot A = L_k^{-1}\cdots L_1^{-1}\cdot D\cdot U_r^{-1}\cdots U_1^{-1}
  \tag{$*{*}*$}
\end{equation*}

A product of lower triangular matrices is lower-triangular,
and a product of upper triangular matrices is 
upper-triangular
so you can combine all the $L$'s and all the $U$'s, and
the formula~($*{*}*$) is often known as $PA=LDU$.

We illustrate with the generic $\nbyn{2}$ transformation of $\Re^2$ represented 
with respect to the standard basis in this way.
\begin{equation*}
  T=
  \begin{mat}
    1 &2 \\
    3 &4
  \end{mat}
\end{equation*}
Gauss's Method is straightforward.
\begin{equation*}
  \begin{mat}
    1 &2 \\
    3 &4
  \end{mat}
  \grstep{-3\rho_1+\rho_2}  
  \begin{mat}
    1 &2 \\
    0 &-2
  \end{mat}
  \grstep{-(1/2)\rho_2}  
  \begin{mat}
    1 &2 \\
    0 &1
  \end{mat}
  \grstep{-2\chi_1+\chi_2}  
  \begin{mat}
    1 &0 \\
    0 &1
  \end{mat}
\end{equation*}
(We use $\chi_i$ for the columns.)
This is the associated factorization.
\begin{equation*}
  \begin{mat}
    1 &2 \\
    3 &4
  \end{mat}
  =
  \begin{mat}
   1 &0 \\
   3 &1 
  \end{mat}
  \begin{mat}
    1 &0 \\
    0 &-2
  \end{mat}
  \begin{mat}
    1 &2 \\
    0 &1
  \end{mat}
\end{equation*}
The product checks out.
\begin{sageoutput}
L = matrix(QQ, [[1, 0], [3, 1]])
D = matrix(QQ, [[1, 0], [0, -2]])
U = matrix(QQ, [[1, 2], [0, 1]])
L*D*U  
\end{sageoutput}

We got into this to understand 
the geometric effect of the generic transformation.
\begin{equation*}
  \rowvec{x &y}
  \begin{mat}
   1 &2 \\
   3 &4 
  \end{mat}
\end{equation*}
\begin{sageoutput}[d,0,4;d,5,7]
load "plot_action.sage"
q = plot_square_action(1,0,0,1) 
q.set_axes_range(-4, 4, -1, 6) 
q.save("graphics/geo300a.png")
p = plot_square_action(1,2,3,4) 
p.set_axes_range(-4, 4, -1, 6) 
p.save("graphics/geo300b.png")
\end{sageoutput}
\begin{sagesilent}
load "plot_action.sage"
q = plot_square_action(1,0,0,1) 
q.set_axes_range(-4, 4, -1, 6) 
q.save("graphics/geo300a.png",dpi=1200)
p = plot_square_action(1,2,3,4) 
p.set_axes_range(-4, 4, -1, 6) 
p.save("graphics/geo300b.png",dpi=1200)
\end{sagesilent}
\begin{equation*}
  \vcenteredhbox{\includegraphics{graphics/geo300a.png}}
  \quad\mapsvia{\big (\begin{smallmatrix} 1 &0 \\ 3 &1 \end{smallmatrix}\big )}\quad
  \vcenteredhbox{\includegraphics{graphics/geo300b.png}}
\end{equation*}
\noindent Expand it using the above $LDU$~factorization. 
\begin{equation*}
  \rowvec{x &y}
  \begin{mat}
   1 &0 \\
   3 &1 
  \end{mat}
  \begin{mat}
    1 &0 \\
    0 &-2
  \end{mat}
  \begin{mat}
    1 &2 \\
    0 &1
  \end{mat}  
\end{equation*}
We will look twice at the effects these three.
First we will see the effect of each of $L$, $D$, and~$U$ separately.
Then we will look at the cumulative effect: that of~$L$, then of~$LD$, 
and finally~$LDU$.

The first matrix is a skew parallel to the $x$-axis.
\begin{sageoutput}[d,0,4;d,5,7]
load "plot_action.sage"
q = plot_square_action(1,0,0,1) 
q.set_axes_range(-4, 4, -1, 4) 
q.save("graphics/geo301a.png")
p = plot_square_action(1,0,3,1) 
p.set_axes_range(-4, 4, -1, 4) 
p.save("graphics/geo301b.png")
\end{sageoutput}
\begin{sagesilent}
load "plot_action.sage"
q = plot_square_action(1,0,0,1) 
q.set_axes_range(-4, 4, -1, 4) 
q.save("graphics/geo301a.png",dpi=1200)
p = plot_square_action(1,0,3,1) 
p.set_axes_range(-4, 4, -1, 4) 
p.save("graphics/geo301b.png",dpi=1200)
\end{sagesilent}
\begin{equation*}
  \vcenteredhbox{\includegraphics{graphics/geo301a.png}}
  \quad\mapsvia{\big (\begin{smallmatrix} 1 &0 \\ 3 &1 \end{smallmatrix}\big )}\quad
  \vcenteredhbox{\includegraphics{graphics/geo301b.png}}
\end{equation*}
\noindent
The second matrix rescales and changes the orientation.
\begin{sageoutput}[d,0,4;d,5,7]
load "plot_action.sage"
q = plot_square_action(1,0,0,1) 
q.set_axes_range(-4, 4, -3, 2) 
q.save("graphics/geo302a.png")
p = plot_square_action(1,0,0,-2) 
p.set_axes_range(-4, 4, -3, 2) 
p.save("graphics/geo302b.png")
\end{sageoutput}
\begin{sagesilent}
load "plot_action.sage"
q = plot_square_action(1,0,0,1) 
q.set_axes_range(-4, 4, -3, 2) 
q.save("graphics/geo302a.png",dpi=1200)
p = plot_square_action(1,0,0,-2) 
p.set_axes_range(-4, 4, -3, 2) 
p.save("graphics/geo302b.png",dpi=1200)
\end{sagesilent}
\begin{equation*}
  \vcenteredhbox{\includegraphics{graphics/geo302a.png}}
  \quad\mapsvia{\big (\begin{smallmatrix} 1 &0 \\ 0 &-2 \end{smallmatrix}\big )}\quad
  \vcenteredhbox{\includegraphics{graphics/geo302b.png}}
\end{equation*}
\noindent
Finally, the third matrix is a skew parallel to the $y$-axis.
\begin{sageoutput}[d,0,4;d,5,7]
load "plot_action.sage"
q = plot_square_action(1,0,0,1) 
q.set_axes_range(-4, 4, -1, 4) 
q.save("graphics/geo303a.png")
p = plot_square_action(1,2,0,1) 
p.set_axes_range(-4, 4, -1, 4) 
p.save("graphics/geo303b.png")
\end{sageoutput}
\begin{sagesilent}
load "plot_action.sage"
q = plot_square_action(1,0,0,1) 
q.set_axes_range(-4, 4, -1, 4) 
q.save("graphics/geo303a.png")
p = plot_square_action(1,2,0,1) 
p.set_axes_range(-4, 4, -1, 4) 
p.save("graphics/geo303b.png")
\end{sagesilent}
\begin{equation*}
  \vcenteredhbox{\includegraphics{graphics/geo303a.png}}
  \quad\mapsvia{\big (\begin{smallmatrix} 1 &2 \\ 0 &1 \end{smallmatrix}\big )}\quad
  \vcenteredhbox{\includegraphics{graphics/geo303b.png}}
\end{equation*}
  
The lower-triangular and upper-triangular matrices do not
change orientation.
Any orientation changing in $LDU$ happens via diagonal entries that are 
negative.
(\textit{Comment.}
However, if a matrix requires row swaps~$PA=LDU$ then the situation is 
more subtle.
Each row swap toggles the orientation, from counterclockwise to clockwise or
from clockwise to counterclockwise.
Thus
if the permutation matrix requires an odd number of swaps 
then it changes the orientation 
but with an even number of swaps it leaves the orientation the same.)

Now we look at the cumulative effect: the actions of $L$, then~$LD$, then~$LDU$.
For example, here is the cumulative effect of the maps on the 
unit square's upper right corner.
\begin{equation*}
  \colvec{1 \\ 1}
  \mapsunder{L}
  \colvec{4 \\ 1}
  \mapsunder{D}
  \colvec{4 \\ -2}
  \mapsunder{U}
  \colvec{4 \\ 6}
\end{equation*}

\begin{sageoutput}[d,5,6;d,7,9;d,10,12;d,13,15;d,16,18]
L = matrix(QQ, [[1, 0], [3, 1]])
D = matrix(QQ, [[1, 0], [0, -2]])
U = matrix(QQ, [[1, 2], [0, 1]])
LD = L*D
LDU = LD*U  
load "plot_action.sage"
p = plot_square_action(1,0,0,1) 
p.set_axes_range(-4, 4, -2, 6) 
p.save("graphics/geo304a.png")
p = plot_square_action(L[0][0], L[0][1], L[1][0], L[1][1]) 
p.set_axes_range(-4, 4, -2, 6) 
p.save("graphics/geo304b.png")
p = plot_square_action(LD[0][0], LD[0][1], LD[1][0], LD[1][1]) 
p.set_axes_range(-4, 4, -2, 6) 
p.save("graphics/geo304c.png")
p = plot_square_action(LDU[0][0], LDU[0][1], LDU[1][0], LDU[1][1]) 
p.set_axes_range(-4, 4, -2, 6) 
p.save("graphics/geo304d.png")
\end{sageoutput}
\begin{sagesilent}
L = matrix(QQ, [[1, 0], [3, 1]])
D = matrix(QQ, [[1, 0], [0, -2]])
U = matrix(QQ, [[1, 2], [0, 1]])
LD = L*D
LDU = LD*U  
load "plot_action.sage"
p = plot_square_action(1,0,0,1) 
p.set_axes_range(-4, 4, -2, 6) 
p.save("graphics/geo304a.png",dpi=1200)
p = plot_square_action(L[0][0], L[0][1], L[1][0], L[1][1]) 
p.set_axes_range(-4, 4, -2, 6) 
p.save("graphics/geo304b.png",dpi=1200)
p = plot_square_action(LD[0][0], LD[0][1], LD[1][0], LD[1][1]) 
p.set_axes_range(-4, 4, -2, 6) 
p.save("graphics/geo304c.png",dpi=1200)
p = plot_square_action(LDU[0][0], LDU[0][1], LDU[1][0], LDU[1][1]) 
p.set_axes_range(-4, 4, -2, 6) 
p.save("graphics/geo304d.png",dpi=1200)
\end{sagesilent}
\begin{center}
  \begin{tabular}{rcl}
    \vcenteredhbox{\includegraphics{graphics/geo304a.png}}
    &$\mapsvia{\big (\begin{smallmatrix} 1 &0 \\ 3 &1 \end{smallmatrix}\big )}$
    &\vcenteredhbox{\includegraphics{graphics/geo304b.png}}  \\
    &$\mapsvia{\big (\begin{smallmatrix} 1 &0 \\ 0 &-2 \end{smallmatrix}\big )}$
    &\vcenteredhbox{\includegraphics{graphics/geo304c.png}}  \\
    &$\mapsvia{\big (\begin{smallmatrix} 1 &2 \\ 0 &1 \end{smallmatrix}\big )}$
    &\vcenteredhbox{\includegraphics{graphics/geo304d.png}} 
  \end{tabular} 
\end{center}



\section{Source of plot\_action.sage}
The \inlinecode{plot_square_action}
routine takes the four entries of the $\nbyn{2}$
matrix and returns a list of graphics.
Most of the heavy lifting here
is done by the helper \inlinecode{color_square_list}, 
given below.
\lstinputlisting[firstline=91,lastline=102]{plot_action.sage}

There are two technical points about the helper routine that could do with 
explanation. 
First is \inlinecode{ZORDER}, which determines the order in which 
\Sage{} plots things, and here we want the unit square to be plotted
after the axes (so its colors will be visible).
The second is that the way line segments butt against each other is ugly 
so we cover the butt with a dot.
\lstinputlisting[firstline=52,lastline=89]{plot_action.sage}


\endinput


TODO:
