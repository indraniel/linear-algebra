\chapter{Geometry of Linear Maps}

\Sage{} can illustrate the geometric effect of linear transformations.
The graphis here picture transformations of the plane~$\Re^2$
because those fit on the paper but the principles extend to 
higher-dimensional spaces.




%========================================
\section{Lines map to lines}
The pictures in this chapter are 
based on the observation that under a linear map, the image of a line
in the domain is a line in the range.

To verify that,
consider a domain space~$\Re^d$ and codomain space~$\Re^c$, along with
the linear map~$h$.
We get a line in the domain by fixing a vector of slopes~$\vec{m}\in\Re^d$
and a vector of offsets from the origin~$\vec{b}\in\Re^d$, 
and then considering the set 
$\ell = \set{\vec{v}=\vec{m}\cdot s +\vec{b}\suchthat s\in\Re}$.
The image of~$\ell$ is this set. 
\begin{equation*}
  h(\ell)=\set{h(\vec{m}\cdot s+\vec{b})\suchthat s\in\Re}
  =\set{h(\vec{m})\cdot s+h(\vec{b})\suchthat s\in\Re}
\end{equation*}
This is a line in the codomain~$\Re^c$ with the vector of 
slopes~$h(\vec{m})$ and the vector of offsets~$h(\vec{b})$. 

For example, consider the transformation $\map{t}{\Re^2}{\Re^2}$ 
that rotates vectors counterclockwise by $\pi/6$~radians.
\begin{equation*}
  \rep{t}{\stdbasis_2,\stdbasis_2}
  =
  \begin{mat}
    \cos(\pi/6)  &\sin(\pi/6) \\
    -\sin(\pi/6)  &\cos(\pi/6)
  \end{mat}
  = 
  \begin{mat}
    \sqrt{3}/2   &1/2 \\
    -1/2          &\sqrt{3}/2
  \end{mat}
\end{equation*}
Also consider the line $y=3x+2$, described here as a set of vectors.
\begin{equation*}
  \ell=\set{\colvec{x \\ y}=\colvec{3 \\ 1}\cdot s+\colvec{2 \\ 0}\suchthat s\in\Re}
\end{equation*}
That line when rotated by~$t$ is this set.
\begin{equation*}
  t(\ell)
  =
  \set{\colvec{x \\ y}=\frac{1}{2}\colvec{3\sqrt{3}-1 \\ 3+\sqrt{3}}\cdot s
                                  +\colvec{\sqrt{3} \\ 1}\suchthat s\in\Re}
\end{equation*}
\begin{sageoutput}
s = var('s')
plot.options['figsize'] = 2.5
plot.options['axes_pad'] = 0.05
plot.options['fontsize'] = 7
plot.options['dpi'] = 500
plot.options['aspect_ratio'] = 1
ell = parametric_plot((3*s+2,1*s), (s, -10, 10))
ell.set_axes_range(-4, 4, -4, 4)
ell.save("sageoutput/plot_action0.png", fontsize=7)
t_x(s) = ((3*sqrt(3)-1)/2)*s+sqrt(3)
t_y(s) = ((3+sqrt(3))/2)*s+1
t_ell = parametric_plot((t_x(s), t_y(s)), (s, -10, 10))
t_ell.set_axes_range(-4, 4, -4, 4)
t_ell.save("sageoutput/plot_action0a.png", fontsize=7)
\end{sageoutput}
\begin{equation*}
  \vcenteredhbox{\includegraphics{sageoutput/plot_action0.png}}
  \quad\mapsvia{t}\quad
  \vcenteredhbox{\includegraphics{sageoutput/plot_action0a.png}}
\end{equation*}
(The limits on the parameter~$s$ of $10$ and~$-10$ are arbitrary, just
chosen to be large enough that the line segment covers the entire 
domain and codomain intervals shown, from $-4$ to~$4$.)




%========================================
\section{The unit square}
We first illustrate the effect of transformations 
$\map{t}{\Re^2}{\Re^2}$ 
by applying them to
this unit square.
\begin{center}
  \includegraphics{sageoutput/plot_action1.png}
\end{center}
That was generated by this \Sage{} session.
\begin{sageoutput}
load "plot_action.sage"
p = plot_square_action(1,0,0,1)  # identity matrix
p.set_axes_range(-0.25, 1.5, -0.25, 1.5) 
p.save("sageoutput/plot_action1.png")
\end{sageoutput}
\noindent The \inlinecode{plot_square_action} 
routine plots the action on the square of a matrix representing the 
transformation.
\begin{equation*}
  % \rep{t}{\stdbasis_2,\stdbasis_2}=
  \rowvec{x  &y}
  \begin{mat}
    a &b \\
    c &d
  \end{mat}
  =
  \rowvec{ax+cy &bx+dy}
\end{equation*}
The routine's source is at the end of this chapter but
the observation above  
that linear maps send lines to lines makes it easy:
the matrix 
has this action on the four corners of the square 
\begin{equation*}
  \colvec{0 \\ 0}\mapsunder{t}\colvec{0 \\ 0}\quad
  \colvec{1 \\ 0}\mapsunder{t}\colvec{a \\ b}\quad
  \colvec{1 \\ 1}\mapsunder{t}\colvec{a+c \\ b+d}\quad
  \colvec{0 \\ 1}\mapsunder{t}\colvec{c \\ d}
\end{equation*}
and the routine plots the four line segments.
In the \Sage{} session above that routine is given the identity matrix,
so it plots the unit square unchanged.

This \Sage{} session
\begin{sageoutput}
load "plot_action.sage"
q = plot_square_action(1,0,0,1) 
q.set_axes_range(-6, 6, -1, 6) 
q.save("sageoutput/plot_action2.png")
p = plot_square_action(1,2,3,4) 
p.set_axes_range(-6, 6, -1, 6) 
p.save("sageoutput/plot_action2a.png")
\end{sageoutput}
\noindent generates these before and 
after pictures of the effect of the generic  
matrix.\footnote{The remaining examples in this chapter omit the 
fiddly lines that load, save, set the axis ranges, etc.}
\begin{equation*}
  \vcenteredhbox{\includegraphics{sageoutput/plot_action2.png}}
  \quad\mapsvia{\big (\begin{smallmatrix} 1 &2 \\ 3 &4 \end{smallmatrix}\big )}\quad
  \vcenteredhbox{\includegraphics{sageoutput/plot_action2a.png}}
\end{equation*}
The colors are there to show that transformations can change
orientations.
Take the colors in their natural order of red, orange, 
green, and blue.
Then the domain square is a counterclockwise shape, while the transformed
square is clockwise.

We can develop an understanding of complex actions by 
building from simple ones.
This transformation doubles the $x$~components of all vectors. 
\begin{sageoutput}[d,0,4;d,5,7]
load "plot_action.sage"
q = plot_square_action(1,0,0,1) 
q.set_axes_range(-4, 4, -1, 4) 
q.save("sageoutput/plot_action3.png")
p = plot_square_action(2,0,0,1) 
p.set_axes_range(-4, 4, -1, 4) 
p.save("sageoutput/plot_action3a.png")
\end{sageoutput}
\begin{equation*}
  \vcenteredhbox{\includegraphics{sageoutput/plot_action3.png}}
  \quad\mapsvia{\big (\begin{smallmatrix} 2 &0 \\ 0 &1 \end{smallmatrix}\big )}\quad
  \vcenteredhbox{\includegraphics{sageoutput/plot_action3a.png}}
\end{equation*}
\noindent
Linear maps send the zero vector to the zero vector so 
the new shape is also anchored at the origin.
But it has been stretched horizontally: it has the same orientation
as the starting square, but twice the area.

If we triple the $x$~coordinate then we get a similar shape but with
three times the area of the starting one.
What about if we take $-1$~times the x-coordinate?
\begin{sageoutput}[d,0,4;d,5,7]
load "plot_action.sage"
q = plot_square_action(1,0,0,1) 
q.set_axes_range(-4, 4, -1, 4) 
q.save("sageoutput/plot_action4.png")
p = plot_square_action(-1,0,0,1) 
p.set_axes_range(-4, 4, -1, 4) 
p.save("sageoutput/plot_action4a.png")
\end{sageoutput}
\begin{equation*}
  \vcenteredhbox{\includegraphics{sageoutput/plot_action4.png}}
  \quad\mapsvia{\big (\begin{smallmatrix} -1 &0 \\ 0 &1 \end{smallmatrix}\big )}\quad
  \vcenteredhbox{\includegraphics{sageoutput/plot_action4a.png}}
\end{equation*}
\noindent
It changes the orientation.
We say that the above shape has an \textit{oriented area}
of $-1$.
The motivation for the sign is:~we start with the right-hand figure 
from the example before this one
and slide the orange side in from the right, from $2$ to~$1$, to~$0$ and
then to~$-1$.
The area falls from~$2$ to~$1$, to~$0$, and it is in some way natural
to assign the figure above an area measure of~$-1$.
The prefix `oriented' is just there to distinguish this idea from the
usual idea of 
area.\footnote{The oriented area is the same as the 
\protect\textit{determinant} of the matrix, which the book covers in the
Chapter Four.}
The usual area is the absolute value of the oriented area.

The next transformation combines action in two axes, 
tripling the $y$~components and multiplying 
$x$~components by $-1$. 
\begin{sageoutput}[d,0,4;d,5,7]
load "plot_action.sage"
q = plot_square_action(1,0,0,1) 
q.set_axes_range(-4, 4, -1, 4) 
q.save("sageoutput/plot_action5.png")
p = plot_square_action(-1,0,0,3) 
p.set_axes_range(-4, 4, -1, 4) 
p.save("sageoutput/plot_action5a.png")
\end{sageoutput}
\begin{equation*}
  \vcenteredhbox{\includegraphics{sageoutput/plot_action5.png}}
  \quad\mapsvia{\big (\begin{smallmatrix} -1 &0 \\ 0 &3 \end{smallmatrix}\big )}\quad
  \vcenteredhbox{\includegraphics{sageoutput/plot_action5a.png}}
\end{equation*}
The colors show that this transformation also changes
the orientation, so the new shape has an oriented area of~$-3$.

What if we change the orientation twice?
\begin{sageoutput}[d,0,4;d,5,7]
load "plot_action.sage"
q = plot_square_action(1,0,0,1) 
q.set_axes_range(-4, 4, -4, 2) 
q.save("sageoutput/plot_action6.png")
p = plot_square_action(-2,0,0,-3) 
p.set_axes_range(-4, 4, -4, 2) 
p.save("sageoutput/plot_action6a.png")
\end{sageoutput}
\begin{equation*}
  \vcenteredhbox{\includegraphics{sageoutput/plot_action6.png}}
  \quad\mapsvia{\big (\begin{smallmatrix} -2 &0 \\ 0 &-3 \end{smallmatrix}\big )}\quad
  \vcenteredhbox{\includegraphics{sageoutput/plot_action6a.png}}
\end{equation*}
The colors are the same as the original shape's counterclockwise
red, orange, green, and blue.
Thus the new shape has an oriented area of~$6$.

We next show the effect of putting in off-diagonal entries.
\begin{sageoutput}[d,0,4;d,5,7]
load "plot_action.sage"
q = plot_square_action(1,0,0,1) 
q.set_axes_range(-4, 4, -1, 4) 
q.save("sageoutput/plot_action7.png")
p = plot_square_action(1,0,2,1) 
p.set_axes_range(-4, 4, -1, 4) 
p.save("sageoutput/plot_action7a.png")
\end{sageoutput}
\begin{equation*}
  \vcenteredhbox{\includegraphics{sageoutput/plot_action7.png}}
  \quad\mapsvia{\big (\begin{smallmatrix} 1 &0 \\ 2 &1 \end{smallmatrix}\big )}\quad
  \vcenteredhbox{\includegraphics{sageoutput/plot_action7a.png}}
\end{equation*}
This transformation is a \textit{skew}\footnote{The word skew means ``to distort from a symmetrical form.''}
(or \textit{shear}).
The line segment sides of the original square 
map to line segments, but these sides are not at right angles.
The action
\begin{equation*}
  \colvec{x \\ y} \mapsto \colvec{x+2y \\ y}
\end{equation*}
means that 
a vector with a $y$~component of~$1$ is shifted right by~$2$ while
a vector with a $y$~component of~$2$ is shifted right by~$4$, so
vectors are shifted depending on how far they are above or below the
$x$-axis.
This transformation preserves orientation and the shape has a base of~$1$
with a height of~$1$ so its oriented area is~$1$.

Putting a nonzero value in the other off-diagonal entry,
the upper right, has much the same effect except that it skews parallel
to the $y$-axis.
\begin{sageoutput}[d,0,4;d,5,7]
load "plot_action.sage"
q = plot_square_action(1,0,0,1) 
q.set_axes_range(-4, 4, -3, 2) 
q.save("sageoutput/plot_action8.png")
p = plot_square_action(1,-2,0,1) 
p.set_axes_range(-4, 4, -3, 2) 
p.save("sageoutput/plot_action8a.png")
\end{sageoutput}
\begin{equation*}
  \vcenteredhbox{\includegraphics{sageoutput/plot_action8.png}}
  \quad\mapsvia{\big (\begin{smallmatrix} 1 &-2 \\ 0 &1 \end{smallmatrix}\big )}\quad
  \vcenteredhbox{\includegraphics{sageoutput/plot_action8a.png}}
\end{equation*}
The action
\begin{equation*}
  \colvec{x \\ y} \mapsto \colvec{x \\ -2x+y}
\end{equation*}
means that vectors are shifted depending on how far they are from the
$x$~axis.
For instance, a vector with an $x$~component of~$1$ is shifted by~$-2$.
The oriented area of this shape is~$1$.




\subsection{Turing's factorization PA=LDU}
We will now see how the action of any matrix can be decomposed into 
the actions shown in the prior section.
This will give us a complete geometric description of any linear map.

Recall that the row operations of Gauss's Method can be done with
matrix multiplication.
For instance, multiplication from the left by this matrix has the effect of the
row operation $2\rho_1+\rho_2$.
\begin{equation*}
  \begin{mat}
    1 &0 &0 \\
    2 &1 &0 \\
    0 &0 &1
  \end{mat}
  \begin{mat}
    3 &1 &4 \\
   -6 &1 &-8 \\
    0 &-3 &2
  \end{mat}
  =
  \begin{mat}
    3 &1  &4 \\ 
    0 &3 &0 \\
    0 &-3  &2
  \end{mat}
\end{equation*}

In general, as described in the book, the  
\definend{elementary reduction matrices}
come in three types $M_i(k)$, $P_{i,j}$, and~$C_{i,j}(k)$, and
arise from applying a row operation to an identity matrix.
\begin{center}
$I\grstep{k\rho_i}M_i(k)$ for \( k\neq 0 \)
\qquad
\( I\grstep{\rho_i\leftrightarrow\rho_j}P_{i,j} \) for \( i\neq j \)
\qquad
\( I\grstep{k\rho_i+\rho_j}C_{i,j}(k) \) for \( i\neq j \)
\end{center}
With these we can use matrix multiplication to do Gauss's Methd:
for a matrix~$H$ we can do row scaling~\( k\rho_i \) 
with \( M_i(k)H \), 
we can swap rows \( \rho_i\leftrightarrow\rho_j \) with \( P_{i,j}H \), 
and we can add a multiple of one row to another
\( k\rho_i+\rho_j \) with \( C_{i,j}(k)H \). 
The prior paragraph used the $\nbyn{3}$
matrix $C_{1,2}(2)$.

We can continue the process from the equation above
and use $C_{2,3}(-1)$ to perform $-\rho_2+\rho_3$,
producing echelon form.
\begin{equation*}
  \begin{mat}
    1 &0  &0 \\
    0 &1  &0 \\
    0 &-1 &1
  \end{mat}
  \begin{mat}
    1 &0 &0 \\
    2 &1 &0 \\
    0 &0 &1
  \end{mat}
  \begin{mat}
    3 &1 &4 \\
   -6 &1 &-8 \\
    0 &-3 &2
  \end{mat}
  =
  \begin{mat}
    3 &1  &4 \\ 
    0 &3  &0 \\
    0 &0  &2
  \end{mat}
  \tag{$*$}
\end{equation*}
As in this example, using matrix multiplication alone
we can do Gauss's Method and produce echelon form.
One way to express this is that
equation~($*$) is a factorization of the starting matrix into a product of
some elementary matrices and an echelon form matrix.  

Observe further that if the starting matrix is such that
we don't need to do any row swapping then 
we can stick elementary
operations that involve only using a row to work on a row below it,
that is,
\( k\rho_i+\rho_j \) where $j>i$. 
The elementary
matrices that perform those operations are \textit{lower triangular}
since all of their nonzero entries are in the lower left.
(Matrices with all of their nonzero entries in the upper right are 
\textit{upper triangular}.)
Thus our factorization so far is into a product of lower triangular elementary
matrices and an echelon form matrix.

We can factor further.
We can use a diagonal matrix to ensure that the 
leading entries of the nonzero rows of the echelon form matrix are~$1$'s.
Here we've done that additional step to equation~($*$).
\begin{equation*}
  \begin{mat}
    1/3 &0   &0 \\
    0   &1/3 &0 \\
    0   &0   &1/2  
  \end{mat}
  \begin{mat}
    1 &0  &0 \\
    0 &1  &0 \\
    0 &-1 &1
  \end{mat}
  \begin{mat}
    1 &0 &0 \\
    2 &1 &0 \\
    0 &0 &1
  \end{mat}
  \begin{mat}
    3 &1 &4 \\
   -6 &1 &-8 \\
    0 &-3 &2
  \end{mat}
  =
  \begin{mat}
    1 &1/3  &4/3 \\ 
    0 &1  &0 \\
    0 &0  &1
  \end{mat}
  \tag{$**$}
\end{equation*}
 
We can use matrix multiplication to factor all the way to a 
block partial identity
matrix.
The idea is to use column operations.
Here we use right-multiplication on the right-hand side of~($**$) to 
add $-1/3$ times the first column to the second column.
\begin{equation*}
  \begin{mat}
    1 &1/3  &4/3 \\ 
    0 &1  &0 \\
    0 &0  &1
  \end{mat}
  \begin{mat}
    1  &-1/3  &0  \\
    0  &1     &0  \\
    0  &0     &1
  \end{mat}
  =
  \begin{mat}
    1 &0  &4 \\ 
    0 &1  &0 \\
    0 &0  &1
  \end{mat}
\end{equation*}
Adding $-4/3$~times the first column to the third column
leave us with an identity matrix.
\begin{equation*}
  \begin{mat}
    3 &1  &4 \\ 
    0 &3  &0 \\
    0 &0  &2
  \end{mat}
  \begin{mat}
    1  &0     &-4/3  \\
    0  &1     &0  \\
    0  &0     &1
  \end{mat}
  =
  \begin{mat}
    1 &0  &0 \\ 
    0 &1  &0 \\
    0 &0  &1
  \end{mat}
   \tag{$*{*}*$}
\end{equation*}
Thus, if we start with a matrix $A$ that does not require any row swaps, 
we get this matrix equation.
\begin{equation*}
  L_1L_2\cdots L_k\cdot A\cdot U_1U_2\cdot U_r = D
\end{equation*}
Here $D$ is a partial identity matrix, 
the $L_i$ are lower-triangular row combination matrices,
and the $U_j$ are upper-triangular column combination matrices. 

We are focused on transformations so we assume all of these
matrices are square.
All of the row operations can be undone (for instance, $2\rho_1+\rho_2$
is undone with $-2\rho_1+\rho_2$), so each of those lower triangular matrices
has an inverse.
Likewise, each of the upper-triangular matrices has an inverse.
Therefore, 
if we don't need any swaps in a Gauss-Jordan reduction of $A$ then we have
this factorization.
\begin{equation*} 
  A = L_k^{-1}\cdots L_1^{-1}\cdot D\cdot U_r^{-1}\cdots U_1^{-1}
\end{equation*}
To fix the swap issue we can pre-swap.
That is, before we factor the starting matrix we can first swap its rows
with a permutation matrix~$P$.
\begin{equation*}
  P\cdot A = L_k^{-1}\cdots L_1^{-1}\cdot D\cdot U_r^{-1}\cdots U_1^{-1}
  \tag{$**$}
\end{equation*}
A product of lower triangular matrices is lower-triangular,
and a product of upper triangular matrices is 
upper-triangular
so we could combine all the $L$'s and all the $U$'s, and
this result is often known as $PA=LDU$.

We illustrate with the generic $\nbyn{2}$ transformation of $\Re^2$ represented 
with respect to the standard basis in this way.
\begin{equation*}
  T=
  \begin{mat}
    1 &2 \\
    3 &4
  \end{mat}
\end{equation*}
Gauss's Method is straightforward.
\begin{equation*}
  \begin{mat}
    1 &2 \\
    3 &4
  \end{mat}
  \grstep{-3\rho_1+\rho_2}  
  \begin{mat}
    1 &2 \\
    0 &-2
  \end{mat}
  \grstep{-(1/2)\rho_2}  
  \begin{mat}
    1 &2 \\
    0 &1
  \end{mat}
  \grstep{-2\chi_1+\chi_2}  
  \begin{mat}
    1 &0 \\
    0 &1
  \end{mat}
\end{equation*}
(We use $\chi_i$ for the columns.)
This is the associated factorization.
\begin{equation*}
  \begin{mat}
    1 &2 \\
    3 &4
  \end{mat}
  =
  \begin{mat}
   1 &0 \\
   3 &1 
  \end{mat}
  \begin{mat}
    1 &0 \\
    0 &-2
  \end{mat}
  \begin{mat}
    1 &2 \\
    0 &1
  \end{mat}
\end{equation*}
The product is as expected.
\begin{sageoutput}
L = matrix(QQ, [[1, 0], [3, 1]])
D = matrix(QQ, [[1, 0], [0, -2]])
U = matrix(QQ, [[1, 2], [0, 1]])
L*D*U  
\end{sageoutput}

We got into this to understand the geometric effect of the transformation.
\begin{equation*}
  \rowvec{x &y}
  \begin{mat}
   1 &2 \\
   3 &4 
  \end{mat}
\end{equation*}
\begin{sageoutput}[d,0,4;d,5,7]
load "plot_action.sage"
q = plot_square_action(1,0,0,1) 
q.set_axes_range(-4, 4, -1, 6) 
q.save("sageoutput/plot_action20.png")
p = plot_square_action(1,2,3,4) 
p.set_axes_range(-4, 4, -1, 6) 
p.save("sageoutput/plot_action20a.png")
\end{sageoutput}
\begin{equation*}
  \vcenteredhbox{\includegraphics{sageoutput/plot_action20.png}}
  \quad\mapsvia{\big (\begin{smallmatrix} 1 &0 \\ 3 &1 \end{smallmatrix}\big )}\quad
  \vcenteredhbox{\includegraphics{sageoutput/plot_action20a.png}}
\end{equation*}
\noindent We expand it according to the above factorization 
\begin{equation*}
  \rowvec{x &y}
  \begin{mat}
   1 &0 \\
   3 &1 
  \end{mat}
  \begin{mat}
    1 &0 \\
    0 &-2
  \end{mat}
  \begin{mat}
    1 &2 \\
    0 &1
  \end{mat}  
\end{equation*}
and look at the effect of the three component matrices.
The first matrix is a skew parallel to the $x$-axis.
\begin{sageoutput}[d,0,4;d,5,7]
load "plot_action.sage"
q = plot_square_action(1,0,0,1) 
q.set_axes_range(-4, 4, -1, 4) 
q.save("sageoutput/plot_action21.png")
p = plot_square_action(1,0,3,1) 
p.set_axes_range(-4, 4, -1, 4) 
p.save("sageoutput/plot_action21a.png")
\end{sageoutput}
\begin{equation*}
  \vcenteredhbox{\includegraphics{sageoutput/plot_action21.png}}
  \quad\mapsvia{\big (\begin{smallmatrix} 1 &0 \\ 3 &1 \end{smallmatrix}\big )}\quad
  \vcenteredhbox{\includegraphics{sageoutput/plot_action21a.png}}
\end{equation*}
\noindent
The second matrix rescales, and changes the orientation.
\begin{sageoutput}[d,0,4;d,5,7]
load "plot_action.sage"
q = plot_square_action(1,0,0,1) 
q.set_axes_range(-4, 4, -3, 2) 
q.save("sageoutput/plot_action22.png")
p = plot_square_action(1,0,0,-2) 
p.set_axes_range(-4, 4, -3, 2) 
p.save("sageoutput/plot_action22a.png")
\end{sageoutput}
\begin{equation*}
  \vcenteredhbox{\includegraphics{sageoutput/plot_action22.png}}
  \quad\mapsvia{\big (\begin{smallmatrix} 1 &0 \\ 0 &-2 \end{smallmatrix}\big )}\quad
  \vcenteredhbox{\includegraphics{sageoutput/plot_action22a.png}}
\end{equation*}
\noindent
Finally, the third matrix is a skew parallel to the $y$-axis.
\begin{sageoutput}[d,0,4;d,5,7]
load "plot_action.sage"
q = plot_square_action(1,0,0,1) 
q.set_axes_range(-4, 4, -1, 4) 
q.save("sageoutput/plot_action23.png")
p = plot_square_action(1,2,0,1) 
p.set_axes_range(-4, 4, -1, 4) 
p.save("sageoutput/plot_action23a.png")
\end{sageoutput}
\begin{equation*}
  \vcenteredhbox{\includegraphics{sageoutput/plot_action23.png}}
  \quad\mapsvia{\big (\begin{smallmatrix} 1 &2 \\ 0 &1 \end{smallmatrix}\big )}\quad
  \vcenteredhbox{\includegraphics{sageoutput/plot_action23a.png}}
\end{equation*}
  

Note that the lower-triangular and upper-triangular matrices do not
change the orientation.
So, absent any row swaps, 
any orientation changing happens when one of the diagonal entries is 
negative, 



\section{Source of plot\_action.sage}
\lstinputlisting{plot_action.sage}

\endinput


TODO:
