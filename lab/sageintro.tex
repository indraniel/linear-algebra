\chapter{\python{} and \sage{}}

You need to learn enough \sage{} to cover the Linear Algebra
in this manual.
For that, you need to understand a bit of \python{}, and an
introduction to \sage.




\section{\python}
\python{} is a popular computer language, often used for scripting,
that is appealing for its simple syntax and powerful libraries.
The significance of `scripting' is that \sage{} uses it in this way,
as a glue to bring together separate parts.

Its home page is \url{http://www.python.org}.
There you can get a download and installation instructions, as well as 
an excellent tutorial that covers much more than we will here.
Here we will acquaint you with enough \python{} to get started, possibly
even if you have no programming experience whatsoever.

First, start \python, for instance by running 
\lstinline[style=inline]!python!
from a command line.
You'll get a couple of lines of 
identifying information followed by three greater-than
characters.
\begin{lstlisting}[style=python]
>>>   
\end{lstlisting}
This is a prompt.
If you type \python{} code and \keyboardkey{Enter} then the system
will read your code, evaluate it, and print the result.
So the prompt lets you try things.
We will see below how to write and run entire \python{} programs
but for the moment we will stick to the experiment. 

Try entering these expressions (double star is exponentiation).
\begin{lstlisting}[style=python]
>>> 2-(-1)
3
>>> 1+2*3
7
>>> 2**3
8  
\end{lstlisting}

Part of \python's appeal is that simple things tend to be simple to do.
Here is how you print something to the screen.
\begin{lstlisting}[style=python]
>>> print 1,"plus",2,"equals",3
1 plus 2 equals 3
\end{lstlisting}
You can often debug just by printing things to the screen at
various steps, and having a straightforward print operator helps. 

\python{} has variables that give you a named place to keep values.
\begin{lstlisting}[style=python]
>>> i=1
>>> i+1
2
\end{lstlisting}
In some programming languages you must declare the `type' of a variable
before you use it; for instance above we would have to tell the system
that $i$ is an integer before we could put the $1$ there.
In contrast, \python{} deduces the type of a variable 
based on what you do to it\Dash above we assigned $1$ to $i$ 
so \python{} deduced that it must be an integer.
Further, we can change how we use the variable and \python{} will 
go along.
\begin{lstlisting}[style=python]
>>> x=1
>>> x
1
>>> x='a'
>>> x
'a'
\end{lstlisting}

\textit{A note on \python{} 3.}
There is a new version of \python{}, called \python~3, with some differences.
For instance, \lstinline[style=inline]!print! works a bit differently.
Here we stick to the older version for a while
because that is what \sage{} does.



\subsection{Basics}
The hash mark \lstinline[style=inline]!#! makes the rest of a line a comment.
\begin{lstlisting}[style=python]
>>> t=2.2
>>> d=(0.5)*9.8*(t**2)  # d in meters
>>> d
23.716000000000005
\end{lstlisting}
(Comments make more sense in a program than at the prompt.)
Programmers often comment an entire line by starting 
that line with a hash. 

As in the listing above, we can can use real 
numbers,\footnote{Technically, floating point numbers.} 
and even complex numbers.
\begin{lstlisting}[style=python]
>>> 5.774*3
17.322
>>> (3+2j)-(1-4j)
(2+6j)
\end{lstlisting}
Notice that \python{} uses `$j$' for the square
root of $-1$, not the `$i$' traditional in mathematics.

The examples above show addition, subtraction, multiplication, 
and exponentiation. 
Division is a bit awkward.
\python{} was originally designed to have the division bar
\lstinline[style=inline]!/! mean real number division when used between
two real numbers (more precisely, when at least one number is real).
However, between two integers the division bar was taken to mean 
what in elementary school is called ``quotient.''
\begin{lstlisting}[style=python]
>>> 5.2/2.0
2.6
>>> 5.2/2
2.6
>>> 5/2
2
\end{lstlisting}
Experience has shown this was a mistake and one of the changes in \python~3
is that the quotient operation will be \lstinline[style=inline]!//!
and the single-bar operator will become real division in all cases.
But for now, the simplest thing to do is to make sure that at least one
number in a division is a real.
\begin{lstlisting}[style=python]
>>> x=5.2
>>> y=2
>>> (1.0*x)/y
2.6
\end{lstlisting}
 


Strings, triple-quoted strings


Dictionary


Lists, Tuples

Booleans, = vs ==,
flow control.

Functions, methods
Define quadratic formula.





\subsection{Objects}

Just as real numbers is a set along with operations, ...

Get today's date.


\subsection{Programs}



%----------------------------------
\section{\sage}

\subsection{Command line}

\subsection{Script}

\subsection{Notebook}

\endinput