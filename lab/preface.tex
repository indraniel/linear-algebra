\chapter*{Preface}\pagestyle{preface}\thispagestyle{preface}

\textit{WARNING! This is an incomplete draft.
It is no doubt riddled with errors.}
\medskip

This collection supplements the text \nocite{Hefferon12}
\textit{Linear Algebra}\footnote{The text's home page 
\protect\url{http://joshua.smcvt.edu/linearalgebra} 
has the PDF, the ancillary materials, and the \protect\LaTeX{} source.}
with a number of explorations that help students
solidify and extend their understanding of the subject, 
using the mathematical software \Sage{}.\footnote{See 
\url{http://www.sagemath.org} for the software and documentation.}

A major goal of any undergraduate Mathematics program is to move students 
toward a higher-level, more abstract, grasp of the subject.
For instance, Calculus classes work on elaborate computations
while later courses spend more effort on concepts and proofs, working
less on the details of calculations.  

The text \textit{Linear Algebra} fits into
this development process.
Naturally it presents the material 
using examples and practice problems
that are small-sized and have manageable numbers:~an 
assignment to multiply a pair of three by three matrices
of small integers will build intuition, whereas asking students to do that same 
by-hand question with twenty by twenty matrices
of ten decimal place numbers would be badgering. 
% (Even more worrying, having students  
% focus their intellectual energy on calculations instead of
% ideas and proofs misleads them as to
% what the subject is about.)

However, an instructor can be concerned that this misses the chance 
to develop the theme that Linear Algebra is very widely applied. 
Mathematical software can mitigate this concern by extending the reach of
what is reasonable 
to bigger systems, harder numbers, and computations 
that\Dash while too much to do by hand\Dash yield
interesting information when they are done by a machine.
This manual extends students's ability to do problems in that way.
For instance, an advantage of learning how to handle these 
tougher computations is that 
they are more like the ones that appear when students apply Linear 
Algebra to other subjects.
Another advantage is that students see new ideas such as 
runtime growth measures.

Well then, why 
not teach straight from the computer system?

Since our goal is to 
develop a higher-level understanding of the material, we want to  
keep the focus on vector spaces and linear maps.
In this exposition 
the computations are a way to develop that understanding, not the main point.

Some instructors may find 
that for their students
the work in this manual is best left aside altogether, 
keeping a tight focus on the
core material.
Other instructors
have students who will benefit from the increased reach that the software
provides.
This manual existence, and status as a separate book, 
gives teachers the freedom
to make the choice that suits their class.


\section{Why \Sage?}
\Sage{} is a very powerful mathematical software systems but so are
many others.
This manual uses it because it is 
Free\footnote{The Free Software Foundation page 
\protect\url{http://www.gnu.org/philosophy/free-sw.html} 
gives background and a definition.} 
and Open Source\footnote{See \protect\url{http://opensource.org/osd.html} 
for a definition.} 
software.

In 
\textit{Open Source Mathematical Software\,}\citep{JoynerStein07}\footnote{See 
\protect\url{http://www.ams.org/notices/200710/tx071001279p.pdf} for the 
full text.}
the authors argue that for Mathematics the best way forward
is to use software that is Open Source.

\begin{quotation}\small
Suppose Jane is a well-known mathematician who announces
she has proved a theorem. We probably will believe
her, but she knows that she will be required to produce
a proof if requested. However, suppose now Jane says a
theorem is true based partly on the results of software. The
closest we can reasonably hope to get to a rigorous proof
(without new ideas) is the open inspection and ability to use
all the computer code on which the result depends. If the
program is proprietary, this is not possible. We have every
right to be distrustful, not only due to a vague distrust of
computers but because even the best programmers regularly
make mistakes.

If one reads the proof of Jane’s theorem in hopes of
extending her ideas or applying them in a new context, it
is limiting to not have access to the inner workings of the
software on which Jane’s result builds.
\end{quotation}  
Professionals choose their tools by balancing many factors but
this argument is persuasive.
We use \Sage{} because it is very capable, 
including at Linear Algebra, so students can 
learn a great deal from it,
and because it is Free.


\section{This manual}
This is Free.
Get the latest version from 
\url{http://joshua.smcvt.edu/linearalgebra}.
Also see that page for the license details and for 
the \LaTeX{} source, including this manual.
I am glad to hear suggestions or corrections, especially from instructors
who have class-tested the material.
My contact information is on the same page. 

The \Sage{} output in this manual was generated automatically so it is 
sure to be
accurate, except that I have (automatically) edited a few lines for
length.
My \Sage{} identifies itself in this way.
\begin{sageoutput}[d,0,1]
version()  
\end{sageoutput}


\section{Acknowledgements}
I am glad for this chance to thank the \Sage{} Development Team for 
their work.
In particular,
without \citep{SageTeam12ref} this manual would not have happened.
I am glad also for the chance to mention 
\citep{Beezer11} as an inspiration.





% \vspace{.25in}
\vfill
\begin{flushright}
\begin{tabular}{l@{}}
Jim Hef{}feron \\
Mathematics, Saint Michael's College \\
Colchester, Vermont USA \\
2012-Sep-10
\end{tabular}  
\end{flushright}

\endinput