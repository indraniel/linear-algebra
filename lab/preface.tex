\chapter*{Preface}\pagestyle{preface}\thispagestyle{preface}


This collection supplements the text
\textit{Linear Algebra}\footnote{\protect\url{http://joshua.smcvt.edu/linearalgebra}}
with a number of computer explorations.

The text develops the material using examples and practice problems
that are small-sized and have simple numbers.
This is natural: while a student must build intuition by
working through the examples and
associated exercises, large or more awkward
cases run the risk of obscuring the issues more than 
clarifying them.
That is, an assignment to multiply two twenty by twenty matrices
of ten decimal place numbers asks a great deal of the patience of the
student, for only a tiny bit more payoff in intuition than they would
get by 
multiplying two three by three matrices of small integers. 

But mathematical software extends our reach to bigger systems and
harder numbers.
One advantage of learning how to process these tougher computations is that 
they are more like the ones that appear when students apply linear 
algebra to in other subjects.

In this manual students will examine the principles and operations 
from the text in the context
of the mathematical software \sage{}.\footnote{\url{http://www.sagemath.org}}


\section{Why \sage?}
\sage{} is a very powerful mathematical software systems but so are
many others.
This manual uses it because it is Free software.

In 
\textit{Open Source Mathematical Software}\footnote{Appeared as an opinion in the \protect\textit{Notices of the American Mathematical Society}, \protect\url{http://www.ams.org/notices/200710/tx071001279p.pdf.}}
the authors argue that for Mathematics, using software that is Open Source
is the right way forward.

\begin{quotation}\small
Suppose Jane is a well-known mathematician who announces
she has proved a theorem. We probably will believe
her, but she knows that she will be required to produce
a proof if requested. However, suppose now Jane says a
theorem is true based partly on the results of software. The
closest we can reasonably hope to get to a rigorous proof
(without new ideas) is the open inspection and ability to use
all the computer code on which the result depends. If the
program is proprietary, this is not possible. We have every
right to be distrustful, not only due to a vague distrust of
computers but because even the best programmers regularly
make mistakes.

If one reads the proof of Jane’s theorem in hopes of
extending her ideas or applying them in a new context, it
is limiting to not have access to the inner workings of the
software on which Jane’s result builds.
\end{quotation}  
While professionals choose their tools by balancing many factors,
I find this argument compelling.
\sage{} is very capable, including at Linear Algebra, and students can 
learn a great deal from it.


\section{This book}
This book is Free and
the latest version is available from its home page 
\url{http://joshua.smcvt.edu/linearalgebra}.
See that page for the license details, and for the \LaTeX{} source.

I am glad to hear suggestions or corrections, especially from instructors
who have class-tested the material.
My contact information is on the same page. 



\vspace{.5in}
\begin{flushright}
\begin{tabular}{l@{}}
Jim Hef{}feron \\
Mathematics, Saint Michael's College \\
2012-Sep-10
\end{tabular}  
\end{flushright}



\endinput