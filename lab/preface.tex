\chapter*{Preface}\pagestyle{preface}\thispagestyle{preface}


This collection supplements the text \nocite{Hefferon12}
\textit{Linear Algebra}\footnote{See \protect\url{http://joshua.smcvt.edu/linearalgebra} for the PDF and \protect\LaTeX{} source.}
with a number of computer explorations that help students
solidify and extend their understanding of the subject.

A major goal of any undergraduate Mathematics program is to,
over time, move students 
toward a higher-level, more abstract, grasp of the subject.
For instance, most Calculus classes work on elaborate computations
while later courses spend more effort on concepts and proofs, allocating 
less attention to the details of calculations.  
The text \textit{Linear Algebra} is designed to fit 
this development process.

Naturally then the text presents its material 
using examples and practice problems
that are small-sized and have manageable numbers:~an 
assignment to multiply a pair of three by three matrices
of small integers will build intuition, whereas asking students to do that same 
by-hand question with twenty by twenty matrices
of ten decimal place numbers would be badgering. 
(And, even more worrisome, requiring that students  
focus their intellectual energy on elaborate calculation instead of directing
their attention to the ideas and proofs would give them the wrong idea of
what the subject is about.)

However, mathematical software can mitigate this by extending the reach of
what is reasonable
to bigger systems, harder numbers, and computations 
that\Dash while too much to do by hand\Dash yield
interesting information when they are done by a machine.
For instance, an advantage of learning how to handle these 
tougher computations is that 
they are more like the ones that appear when students apply Linear 
Algebra to other subjects.
Another advantage is that students see new ideas such as 
runtime growth measures.
In this manual students examine the principles and operations 
from \textit{Linear Algebra} using 
the mathematical software \Sage{}.\footnote{\url{http://www.sagemath.org}}

Well, if computer computations are good then why do by-hand ones at all; why 
not teach straight from the computer system?
Remember that we want to develop a higher-level understanding of the 
material, 
keeping the focus on vector spaces and linear maps.
The computations are a tool to do that, not the main point.
This manual is a separate book because some instructors may find 
that for their students
this work is best left aside altogether, 
while other instructors may have students who
will benefit.


\section{Why \Sage?}
\Sage{} is a very powerful mathematical software systems but so are
many others.
This manual uses it because it is 
Free,\footnote{See \protect\url{http://www.gnu.org/philosophy/free-sw.html} 
for a definition.} 
and Open Source,\footnote{See \protect\url{http://opensource.org/osd.html} 
for a definition.} software.

In 
\textit{Open Source Mathematical Software\,}\citep{JoynerStein07}\footnote{See 
\protect\url{http://www.ams.org/notices/200710/tx071001279p.pdf} for the 
full text.}
the authors argue that for Mathematics using software that is Open Source
is the right way forward.

\begin{quotation}\small
Suppose Jane is a well-known mathematician who announces
she has proved a theorem. We probably will believe
her, but she knows that she will be required to produce
a proof if requested. However, suppose now Jane says a
theorem is true based partly on the results of software. The
closest we can reasonably hope to get to a rigorous proof
(without new ideas) is the open inspection and ability to use
all the computer code on which the result depends. If the
program is proprietary, this is not possible. We have every
right to be distrustful, not only due to a vague distrust of
computers but because even the best programmers regularly
make mistakes.

If one reads the proof of Jane’s theorem in hopes of
extending her ideas or applying them in a new context, it
is limiting to not have access to the inner workings of the
software on which Jane’s result builds.
\end{quotation}  
While professionals choose their tools by balancing many factors,
this argument is persuasive.
This manual uses \Sage{} because it is very capable, 
including at Linear Algebra, because students can 
learn a great deal from it,
and because it is Free.


\section{This book}
This book is Free and
the latest version is available from its home page 
\url{http://joshua.smcvt.edu/linearalgebra}.
See that page for the license details and the \LaTeX{} source.

I am glad to hear suggestions or corrections, especially from instructors
who have class-tested the material.
My contact information is on the same page. 



\vspace{.5in}
\begin{flushright}
\begin{tabular}{l@{}}
Jim Hef{}feron \\
Mathematics, Saint Michael's College \\
2012-Sep-10
\end{tabular}  
\end{flushright}

% \begin{center}
%   \begin{tabular}{lrp{4in}}
%     Concrete Roman  &abcdefghi 
%                          &af a fkla sdf a;lsdklf a ei ja k m,f le aj sd f;la jsdlf kla sd flaie fa j ;kl j3 ;lq kl2 3kl ia ;lf898 lk a;df jl
%   \\
%     Latin Modern mono &{\latinmodernmono abcdefghi} 
%                          &{\latinmodernmono af a fkla sdf a;lsdklf a ei ja k m,f le aj sd f;la jsdlf kla sd flaie fa j ;kl j3 ;lq kl2 3kl ia ;lf898 lk a;df jl}
%   \\
%     TeX Gyre Cursor &{\texgyrecursor abcdefghi} 
%                          &{\texgyrecursor af a fkla sdf a;lsdklf a ei ja k m,f le aj sd f;la jsdlf kla sd flaie fa j ;kl j3 ;lq kl2 3kl ia ;lf898 lk a;df jl}
%   \\
%     Luxi mono &{\luximono abcdefghi} 
%                          &{\luximono af a fkla sdf a;lsdklf a ei ja k m,f le aj sd f;la jsdlf kla sd flaie fa j ;kl j3 ;lq kl2 3kl ia ;lf898 lk a;df jl}
%   \end{tabular}
% \end{center}

\endinput